\documentclass[
]{jss}

%% recommended packages
\usepackage{orcidlink,thumbpdf,lmodern}

\usepackage[utf8]{inputenc}

\author{
Fabien Girka~\orcidlink{0000-0003-2843-1104}\\Laboratoire des Signaux et
Systèmes \And Etienne Camenen\\INSERM \And Caroline Peltier\\INRAE
\AND Arnaud Gloaguen\\CEA \And Vincent Guillemot\\Institut Pasteur
\AND Laurent Le Brusquet\\Laboratoire des Signaux et
Systèmes \And Arthur
Tenenhaus~\orcidlink{0000-0003-3459-1518}\\Laboratoire des Signaux et
Systèmes
}
\title{Multiblock data analysis with the RGCCA package}

\Plainauthor{Fabien Girka, Etienne Camenen, Caroline Peltier, Arnaud
Gloaguen, Vincent Guillemot, Laurent Le Brusquet, Arthur Tenenhaus}
\Plaintitle{Multiblock data analysis with the RGCCA package}
\Shorttitle{Regularized Generalized Canonical Correlation Analysis}


\Abstract{
Multiblock component methods play a central role in exploring complex
relationships between multiple sets of variables. Regularized
generalized canonical correlation analysis (RGCCA) is a unified and
versatile framework that consolidates over six decades of multiblock
component methods. The \pkg{RGCCA} package offers an implementation for
this framework. Beyond implementation, the \pkg{RGCCA} package proposes
graphical outputs and statistics to assess the robustness/significance
of the analysis. The usefulness of the \pkg{RGCCA} package is
illustrated in this paper on two real datasets. The \pkg{RGCCA} package
is freely available on the Comprehensive \proglang{R} Archive Network
(CRAN) \url{http://www.r-project.org/} and GitHub
\url{https://github.com/rgcca-factory/RGCCA}.
}

\Keywords{Multiblock component methods, RGCCA, data integration}
\Plainkeywords{Multiblock component methods, RGCCA, data integration}

%% publication information
%% \Volume{50}
%% \Issue{9}
%% \Month{June}
%% \Year{2012}
%% \Submitdate{}
%% \Acceptdate{2012-06-04}

\Address{
    Fabien Girka\\
    Laboratoire des Signaux et Systèmes\\
    Université Paris-Saclay, CNRS, CentraleSupélec,
\newline  Gif-sur-Yvette, 91190, France \newline Institut du Cerveau de
Paris (ICM), Sorbonne Université \newline  Assistance Publique-Hôpitaux
de Paris (AP-HP)\\
  E-mail: \email{fabien.girka@centralesupelec.fr}\\
  
      Etienne Camenen\\
    INSERM\\
    Inserm U1127, CNRS UMR 7225, \newline Hôpital Universitaire
Pitié-Salpêtrière \newline  75013 Paris, France \newline Institut du
Cerveau de Paris (ICM), Sorbonne Université \newline  Assistance
Publique-Hôpitaux de Paris (AP-HP)\\
  
  
      Caroline Peltier\\
    INRAE\\
    Centre des Sciences du Goût et de l'Alimentation \newline CNRS,
INRAE, Institut Agro \newline University of Bourgogne F-21000 Dijon
\newline CNRS, INRAE, PROBE research infrastructure \newline  ChemoSens
facility, F-21000 Dijon, France \newline Institut du Cerveau de Paris
(ICM), Sorbonne Université \newline  Assistance Publique-Hôpitaux de
Paris (AP-HP)\\
  
  
      Arnaud Gloaguen\\
    CEA\\
    Centre National de Recherche en Génomique Humaine \newline  Institut
de Biologie François Jacob \newline  CEA, Université Paris-Saclay, Évry,
France\\
  
  
      Vincent Guillemot\\
    Institut Pasteur\\
    Université Paris Cité \newline Bioinformatics and Biostatistics Hub
\newline  F-75015 Paris, France\\
  
  
      Laurent Le Brusquet\\
    Laboratoire des Signaux et Systèmes\\
    Université Paris-Saclay, CNRS, CentraleSupélec
\newline  Gif-sur-Yvette, 91190, France\\
  
  
      Arthur Tenenhaus\\
    Laboratoire des Signaux et Systèmes\\
    Université Paris-Saclay, CNRS, CentraleSupélec
\newline  Gif-sur-Yvette, 91190, France \newline Institut du Cerveau de
Paris (ICM), Sorbonne Université \newline  Assistance Publique-Hôpitaux
de Paris (AP-HP)\\
  E-mail: \email{arthur.tenenhaus@centralesupelec.fr}\\
  
  }


% tightlist command for lists without linebreak
\providecommand{\tightlist}{%
  \setlength{\itemsep}{0pt}\setlength{\parskip}{0pt}}

% From pandoc table feature
\usepackage{longtable,booktabs,array}
\usepackage{calc} % for calculating minipage widths
% Correct order of tables after \paragraph or \subparagraph
\usepackage{etoolbox}
\makeatletter
\patchcmd\longtable{\par}{\if@noskipsec\mbox{}\fi\par}{}{}
\makeatother
% Allow footnotes in longtable head/foot
\IfFileExists{footnotehyper.sty}{\usepackage{footnotehyper}}{\usepackage{footnote}}
\makesavenoteenv{longtable}



\usepackage{amsmath} \usepackage{amsfonts} \usepackage{times} \usepackage{bm} \usepackage{soul} \usepackage{epsfig} \usepackage{amssymb} \usepackage{lscape} \usepackage{float} \usepackage{latexsym} \usepackage{graphicx,psfrag,color} \usepackage{multirow}  \usepackage[space]{grffile} \usepackage{amsthm} \usepackage{enumerate} \usepackage{enumitem} \usepackage{setspace} \usepackage{subfigure} \usepackage{longtable} \usepackage{etoolbox}  \usepackage{pdfpages} \usepackage[mathscr]{euscript} \usepackage[T1]{fontenc} \usepackage[misc]{ifsym} \usepackage{wasysym} \usepackage{hyperref} \usepackage[width=\textwidth]{caption} \usepackage{algorithmic, algorithm}

\begin{document}



\newtheorem{theorem}{theorem}[section]%
\newtheorem{lemma}[theorem]{Lemma}
\newtheorem{proposition}[theorem]{Proposition}
\newtheorem{corollary}[theorem]{Corollary}
\newtheorem{remark}[theorem]{Remark}

\section{Introduction}\label{introduction}

A challenging problem in multivariate statistics is to study
relationships between several sets of variables measured on the same set
of individuals. This paradigm is referred to by several names, including
``learning from multimodal data'', ``data integration'', ``multiview
data'', ``multisource data'', ``data fusion'', or ``multiblock data
analysis''. Despite the availability of various statistical methods and
dedicated software for multiblock data analysis, handling multiple
highly multivariate datasets remains a critical issue for effective
analysis and knowledge discovery.

Regularized generalized canonical correlation analysis
\citep[RGCCA,][]{Tenenhaus2011, Tenenhaus2014, Tenenhaus2015, Tenenhaus2017}
is a unified statistical framework that gathers six decades of
multiblock component methods. While RGCCA encompasses a large number of
methods, it is based on a single optimization problem that bears
immediate practical implications, facilitating statistical analyses and
implementation strategies. In this paper, we present the \proglang{R}
package called \pkg{RGCCA}, which implements the RGCCA framework.

For the statistical computing environment \proglang{R} \citep{R2022},
various \proglang{R} packages have come to the fore for carrying out
multiblock data analysis. The bioinformatician community mostly uses
\pkg{mixOmics} \citep{Rohart2017}. It wraps the main functions of the
\pkg{RGCCA} package for performing multiblock analyses.

The \pkg{ade4} package \citep{Dray2007} covers a wide range of
multivariate methods. \pkg{ade4} mainly relies on three approaches for
performing multiblock analysis: multiple co-inertia analysis
\citep[MCOA, ][]{Chessel1996}, multiple factor analysis
\citep[MFA,][]{Escofier1994} and Statis \citep{Lavit1994}.

\pkg{FactoMineR} \citep{Le2008} is also one widely used package for
multiblock methods mainly due to the implementation of MFA and, to a
somewhat lesser extent, generalized Procrustes analysis
\citep[GPA,][]{Gower1975}.

The \pkg{multiblock} package \citep{Liland2022} covers a wide range of
multiblock methods for data fusion but intensively relies on a wrapper
strategy for performing multiblock analyses. Consequently, this results
in strong dependencies between the \pkg{multiblock} package and other
packages including (i) \pkg{FactoMineR} (for MFA and GPA), \pkg{RGCCA}
(for hierarchical PCA \citeauthor{Wold1996} \citeyear{Wold1996};
\citeauthor{Hanafi2010} \citeyear{Hanafi2010}, MCOA, inter-battery
factor analysis \citeauthor{Tucker1958} \citeyear{Tucker1958} and
Carroll's GCCA \citeauthor{Carroll1968a} \citeyear{Carroll1968a}),
\pkg{ade4}
\citep[multiblock redundancy analysis][and Statis]{Bougeard2011},
\pkg{r.jive} \citep[for JIVE][]{Lock2013}, and \pkg{RegularizedSCA}
\citep[for DISCO][]{Schouteden2014}.

To the sake of completeness, we also mention the \pkg{multiview}
\proglang{R} package that can be used for predicting a univariate
response (member of the exponential family of distributions) from
several blocks of variables \citep{Ding2022}.

Each package uses its own way of specifying multiblock models and
storing the results.

For \proglang{Python} users, \pkg{mvlearn} \citep{Perry2021} seems to be
the most advanced \proglang{Python} module for multiview data.
\pkg{mvlearn} offers a suite of algorithms for learning latent space
embeddings and joint representations of views, including a limited
version of the RGCCA framework, a kernel version of GCCA
\citep{Hardoon2004, Bach2002}, and deep CCA \citep{Andrew2013}. Several
other methods for dimensionality reduction and joint subspace learning
are also included, such as multiview multidimensional scaling
\citep{Trendafilov2010}.

From this perspective, it appears that the \pkg{RGCCA} package already
plays a central role within the \proglang{R} community and beyond for
conducting multiblock analyses. Also, special attention has been paid to
ensuring that \pkg{RGCCA}'s implementation of various multiblock
component methods, including MCOA and MFA, aligns with results obtained
from other packages such as \pkg{ade4} or \pkg{FactoMineR}.

Within the \pkg{RGCCA} package, all implemented methods from the
literature are made available through the single \code{rgcca()} function
and thus share the same function interface and a clear class structure.
In addition to the main statistical functionalities, the \pkg{RGCCA}
package provides several utility functions for data preprocessing and
offers various advanced \pkg{ggplot2} \citep{Wickham2016} visualization
tools to help users interpret and extract meaningful information from
integrated data. It also includes metrics for assessing the robustness
and significance of the analysis. The package also includes several
built-in datasets and examples to help users get started quickly.
Package \pkg{RGCCA} is available from the Comprehensive \proglang{R}
Archive Network (CRAN), at
\url{https://CRAN.R-project.org/package=RGCCA} and can be installed from
the \proglang{R} console with the following command:

\footnotesize

\begin{CodeChunk}
\begin{CodeInput}
R> install.packages("RGCCA")
\end{CodeInput}
\end{CodeChunk}

\normalsize

This paper presents an overview of the RGCCA framework's theoretical
foundations, summarizes the optimization problem under which all the
algorithms were designed, and provides code examples to illustrate the
package's usefulness and versatility. Our package offers a valuable
contribution to the field of multiblock data analysis and will enable
researchers to conduct more effective analyses and gain new insights
into complex datasets.

The paper's remaining sections are organized as follows: Section 2
presents the RGCCA framework, how it generalizes existing methods from
the literature, and the master algorithm underlying the RGCCA framework.
Section 3 presents the structure of the package and provides code
examples to illustrate the package's capabilities. Finally, we conclude
the paper in Section 4.

\section{The RGCCA framework}\label{the-rgcca-framework}

We first introduce the optimization problem defining the RGCCA
framework, previously published in
\cite{Tenenhaus2011, Tenenhaus2014, Tenenhaus2015, Tenenhaus2017}. We
then show how it includes many existing methods.

\subsection{Formulation}\label{formulation}

Let \(\boldsymbol x\) be a random column vector of \(p\) variables such
that \(\boldsymbol x\) is the concatenation of \(J\) subvectors
\(\boldsymbol x_1, \dots, \boldsymbol x_J\), with
\(\boldsymbol x_j = (x_{j1}, \ldots, x_{jp_j})^\top\) for
\(j \in \{1, \dots, J\}\). We assume that \(\boldsymbol x\) has zero
mean and a finite second-order moment. Its covariance matrix
\(\mathbf \Sigma\) is then composed of \(J^2\) submatrices:
\(\mathbf \Sigma_{jk} = \mathbb{E}\left[\boldsymbol x_j \boldsymbol x_k^\top\right]\)
for \((j, k) \in \{1, \dots, J\}^2\). Let
\(\mathbf a_j = (a_{j1}, \ldots, a_{jp_j})^\top\) be a non-random
\(p_j\)-dimensional column vector. A composite variable \(y_j\) is
defined as the linear combination of the elements of
\(\boldsymbol x_j\): \(y_j =  \mathbf a_j^\top \boldsymbol x_j\).
Therefore the covariance between two composite variables is
\(\mathbf{a}_j^\top \mathbf \Sigma_{jk} \mathbf{a}_k\).

The RGCCA framework aims to extract the information shared by the \(J\)
random composite variables, taking into account an undirected graph of
connections between them. It consists in maximizing the sum of the
covariances between ``connected'' composites \(y_j\) and \(y_k\) subject
to specific constraints on the weights \(\mathbf a_j\) for
\(j \in \{1, \ldots, J\}\). The following optimization problem thus
defines the RGCCA framework:

\begin{equation}
\underset{\mathbf{a}_1, \ldots,\mathbf{a}_J}{\text{max }}  \displaystyle  \sum_{j, k = 1}^J c_{jk} \text{
g}\left(\mathbf{a}_j^\top  \mathbf \Sigma_{jk} \mathbf{a}_k\right)\\
\text{ s.t. } \mathbf{a}_j \in \Omega_j, j =1, \ldots, J,
\label{optim_RGCCA}
\end{equation} where

\begin{itemize}
\item the set $\Omega_j$ is compact.

\item the function $g$ is any continuously differentiable convex function. Typical choices of $g$ are the identity (horst scheme, leading to maximizing the sum of covariances between block components), the absolute value\footnote{The scheme $g(x) = \vert x \vert$ can be included in this class of functions because the case $x=0$ never appears in practical applications.} (centroid scheme, yielding maximization of the sum of the absolute values of the covariances), the square function (factorial scheme, thereby maximizing the sum of squared covariances), or, more generally, for any even integer $m$, $g(x) = x^m$ (m-scheme, maximizing the power of $m$ of the sum of covariances). The horst scheme penalizes negative structural correlation between block components, while the centroid scheme and the m-scheme enable two components to be negatively correlated. 

\item the design matrix $\mathbf C = \lbrace c_{jk}\rbrace$ is a symmetric $J \times J$ matrix of non-negative elements describing the network of connections between blocks that the user wants to take into account. Usually, $c_{jk} = 1$ between two connected blocks and $0$ otherwise. 
\end{itemize}

\subsection{Sample-based RGCCA}\label{sample-based-rgcca}

A sample-based optimization problem related to \eqref{optim_RGCCA} can
be defined by considering \(n\) observations of the vector
\(\boldsymbol x\). It yields a column-partition data matrix
\(\mathbf X = [\mathbf X_1, \ldots, \mathbf X_j, \ldots, \mathbf X_J] \in \mathbb{R}^{n \times p}\).
Each \(n\times p_j\) data matrix \(\mathbf X_j\) is called a block and
represents a set of \(p_j\) variables observed on the same set of \(n\)
individuals. The variables' number and nature may differ from one block
to another, but the individuals must be the same across blocks. We
assume that all variables are centered.

The \pkg{RGCCA} package proposes a first implementation of the RGCCA
framework, focusing on the following sample-based optimization problem:

\begin{equation}
\underset{\mathbf{a}_1, \ldots,\mathbf{a}_J}{\text{max }}  \displaystyle  \sum_{j, k = 1}^J c_{jk} \text{
g}\left(\mathbf{a}_j^\top  \widehat{\mathbf{\Sigma}}_{jk} \mathbf{a}_k\right)\\
\text{ s.t. } \mathbf a_j^\top \widehat{\mathbf{\Sigma}}_{jj}\mathbf{a}_j = 1, j =1, \ldots, J.
\label{optim_RGCCA_sample1}
\end{equation}

where
\(\widehat{\mathbf{\Sigma}}_{jk}= n^{-1} \mathbf X_j^\top  \mathbf X_k\)
is an estimate of the interblock covariance matrix
\(\mathbf{\Sigma}_{jk}= \mathbb{E}[\boldsymbol x_j\boldsymbol x_k^\top]\)
and \(\widehat{\boldsymbol \Sigma}_{jj}\) is an estimate of the
intra-block covariance matrix
\(\mathbf{\Sigma}_{jj} = \mathbb{E}[\boldsymbol x_j\boldsymbol x_j^\top]\).
As we will see in the next section, an important family of methods is
recovered by choosing
\(\Omega_j = \{\mathbf a_j \in \mathbb{R}^{p_j}; \mathbf a_j^\top \widehat{\boldsymbol \Sigma}_{jj} \mathbf a_j = 1\}\).
In cases involving multi-collinearity within blocks or in high
dimensional settings, one way of obtaining an estimate for the true
covariance matrix \(\mathbf{\Sigma}_{jj}\) is to consider the class of
linear convex combinations of the sample covariance matrix
\(\mathbf{S}_{jj} = n^{-1} \mathbf X_j^\top  \mathbf X_j\) and the
identity matrix \(\mathbf I_{p_j}\). Therefore,
\(\widehat{\mathbf{\Sigma}}_{jj} = (1-\tau_j)\mathbf{S}_{jj} + \tau_j\mathbf{I}_{p_j}\)
with \(\tau_j \in [0,1]\) (shrinkage estimator of
\(\mathbf{\Sigma}_{jj}\)).

From this viewpoint, an equivalent formulation of optimization problem
\eqref{optim_RGCCA_sample1} is given hereafter and enables a better
description of the objective of RGCCA.

\begin{equation}
\underset{\mathbf{a}_1, \ldots,\mathbf{a}_J}{\text{max }} \displaystyle \sum_{j, k = 1}^J c_{jk} \text{g}\left(\widehat{\text{cov}}\left(\mathbf{X}_j\mathbf{a}_j, \mathbf{X}_k\mathbf{a}_k\right)\right) 
\text{ s.t. } (1-\tau_j)\widehat{\text{var}}(\mathbf{X}_j\mathbf{a}_j) + \tau_j \Vert \mathbf{a}_j \Vert^2 = 1, j =1, \ldots, J. 
\label{optim_RGCCA_sample2}
\end{equation}

The objective of RGCCA is thus to find block components
\(\mathbf y_j =  \mathbf X_j  \mathbf a_j\) for
\(j \in \{1, \ldots, J\}\), summarizing the relevant information between
and within the blocks. The \(\tau_j\)s are called shrinkage parameters
ranging from \(0\) to \(1\) and interpolate smoothly between maximizing
the covariance or the correlation. Setting \(\tau_j\) to 0 will force
the block components to unit variance, in which case the covariance
criterion boils down to the correlation. Setting \(\tau_j\) to 1 will
normalize the block weight vectors, which applies the covariance
criterion. A value between \(0\) and \(1\) will lead to a compromise
between the two first options. We can discuss the choice of shrinkage
parameters by providing interpretations of the properties of the
resulting block components:

\begin{itemize}
\item   $\tau_j=1$ is recommended when the user wants a stable component (large variance) while simultaneously taking into account the correlations between blocks. The user must, however, be aware that variance dominates over correlation.

\item   $\tau_j=0$ is recommended when the user wants to maximize correlations between connected components. This option can yield unstable solutions in case of multi-collinearity and cannot be used when a data block is rank deficient (e.g., $n<p_j$).

\item   $0<\tau_j<1$ is a good compromise between variance and correlation: the block components are simultaneously stable and as correlated as possible with their connected block components. This setting can be used when the data block is rank deficient.
\end{itemize}

It is worth pointing out that for each block \(j\), an appropriate
shrinkage parameter \(\tau_j\) can be obtained using various analytical
formulae \citep[see][for instance]{Ledoit2004, Schafer2005, Chen2011}.
As \(\widehat{\mathbf{\Sigma}}_{jj}\) must be positive definite,
\(\tau_j = 0\) can only be selected for a full rank data matrix
\(\mathbf{X}_j\). In the \pkg{RGCCA} package, for each block, the
determination of the shrinkage parameter can be made fully automatic by
using the analytical formula proposed by \cite{Schafer2005} or guided by
the context of an application by cross-validation or permutation.

From the optimization problem (\ref{optim_RGCCA_sample2}), the term
``generalized'' in the acronym of RGCCA embraces at least four notions.
The first one relates to the generalization of two-block methods -
including canonical correlation analysis \citep{Hotelling1936},
inter-battery factor analysis \citep{Tucker1958}, and redundancy
analysis \citep{Wollenberg1977} - to three or more sets of variables.
The second one relates to the ability to take into account some
hypotheses on between-block connections: the user decides which blocks
are connected and which ones are not. The third one relies on the
choices of the shrinkage parameters, allowing the capture of both
correlation or covariance-based criteria. The fourth one relates to the
function \(g\) that enables considering different functions of the
covariance. A triplet of parameters embodies this generalization: (g,
\(\tau_j, \mathbf C\)) and by the fact that an arbitrary number of
blocks can be handled. This triplet of parameters offers flexibility and
allows RGCCA to encompass a large number of multiblock component methods
that have been published for sixty years. Tables
\ref{twoblock_methods}-\ref{multiblock_hierarchical} give the
correspondences between the triplet (g, \(\tau_j, \mathbf C\)) and the
multiblock component methods. For a complete overview, see
\cite{Tenenhaus2017}.

\subsection{Special cases}\label{special-cases}

Two families of methods have come to the fore in the field of multiblock
data analysis. These methods rely on correlation-based or
covariance-based criteria. Canonical correlation analysis
\citep{Hotelling1936} is the seminal paper for the first family, and
Tucker's inter-battery factor analysis \citep{Tucker1958} is for the
second one. These two methods have been extended to more than two blocks
in many ways:

\begin{itemize}
\item
  Main contributions for generalized canonical correlation analysis
  (GCCA) are found in
  \cite{Horst1961, Carroll1968a, Kettenring1971, Wold1982, Wold1985, Hanafi2007}.
\item
  Main contributions for extending Tucker's method to more than two
  blocks come from
  \cite{Carroll1968b, Chessel1996, Hanafi2006, Hanafi2010, Hanafi2011, Hanafi2006, Kramer2007, Smilde2003, TenBerge1988, VandeGeer1984, Westerhuis1998, Wold1982, Wold1985}.
\item
  \cite{Carroll1968b} proposed the ``mixed'' correlation and covariance
  criterion. \cite{Wollenberg1977} combined correlation and variance for
  the two-block situation (redundancy analysis). This method is extended
  to the multiblock situation in \cite{Tenenhaus2011, Tenenhaus2017}.
\end{itemize}

In the two block case, the optimization problem
(\ref{optim_RGCCA_sample2}) reduces to: \begin{equation}
\underset{ \mathbf a_1,  \mathbf a_2}{\text{maximize}} \widehat{\text{
cov}}\left(\mathbf X_1 \mathbf a_1, \mathbf X_2 \mathbf a_2 \right) \text{ s.t. } (1-\tau_j)\widehat{\text{var}}(\mathbf X_j \mathbf a_j) + \tau_j
\Vert  \mathbf a_j \Vert^2 = 1, j =1,2.
\label{rCCA} 
\end{equation}

This problem has been introduced under the name of regularized canonical
correlation analysis \citep{Vinod1976, Leurgans1993, Shawe2004}. For
various extreme cases \(\tau_1 = 0\) or \(1\) and \(\tau_2 = 0\) or
\(1\), optimization problem (\ref{rCCA}) covers a situation which goes
from canonical correlation analysis \citep{Hotelling1933} to Tucker's
inter-battery factor analysis \citep{Tucker1958}, while passing through
redundancy analysis \citep{Wollenberg1977}. This framework corresponds
exactly to the one proposed by \cite{Borga1997} and \cite{Burnham1996}
and is reported in Table \ref{twoblock_methods}.

\begin{longtable}[]{@{}
  >{\raggedright\arraybackslash}p{(\columnwidth - 6\tabcolsep) * \real{0.3750}}
  >{\centering\arraybackslash}p{(\columnwidth - 6\tabcolsep) * \real{0.0982}}
  >{\raggedright\arraybackslash}p{(\columnwidth - 6\tabcolsep) * \real{0.2411}}
  >{\centering\arraybackslash}p{(\columnwidth - 6\tabcolsep) * \real{0.2857}}@{}}
\caption{Two-block component methods.
\label{twoblock_methods}}\tabularnewline
\toprule\noalign{}
\begin{minipage}[b]{\linewidth}\raggedright
\textbf{Methods}
\end{minipage} & \begin{minipage}[b]{\linewidth}\centering
\(g(x)\)
\end{minipage} & \begin{minipage}[b]{\linewidth}\raggedright
\(\tau_j\)
\end{minipage} & \begin{minipage}[b]{\linewidth}\centering
\(\mathbf{C}\)
\end{minipage} \\
\midrule\noalign{}
\endfirsthead
\toprule\noalign{}
\begin{minipage}[b]{\linewidth}\raggedright
\textbf{Methods}
\end{minipage} & \begin{minipage}[b]{\linewidth}\centering
\(g(x)\)
\end{minipage} & \begin{minipage}[b]{\linewidth}\raggedright
\(\tau_j\)
\end{minipage} & \begin{minipage}[b]{\linewidth}\centering
\(\mathbf{C}\)
\end{minipage} \\
\midrule\noalign{}
\endhead
\bottomrule\noalign{}
\endlastfoot
\textbf{Canonical correlation analysis} \citep{Hotelling1936} & \(x\) &
\(\tau_1 = \tau_2 = 0\) &
\(\mathbf{C}_1 = \begin{pmatrix} 0 & 1 \\ 1 & 0 \end{pmatrix}\) \\
\textbf{Inter-battery factor analysis} \citep{Tucker1958} or \textbf{PLS
regression} \citep{Wold1983} & \(x\) & \(\tau_1 = \tau_2 = 1\) &
\(\mathbf{C}_1\) \\
\textbf{Redundancy analysis} \citep{Wollenberg1977} & \(x\) &
\(\tau_1 = 1\) ; \(\tau_2 = 0\) & \(\mathbf{C}_1\) \\
\textbf{Regularized redundancy analysis}
\citep{Takane2007, Bougeard2008, Qannari2005} & \(x\) &
\(0 \le \tau_1 \le 1\) ; \(\tau_2 = 0\) & \(\mathbf{C}_1\) \\
\textbf{Regularized canonical correlation analysis}
\citep{Vinod1976, Leurgans1993, Shawe2004} & \(x\) &
\(0 \le \tau_1 \le 1\) ; \(0 \le \tau_2 \le 1\) & \(\mathbf{C}_1\) \\
\end{longtable}

In the multiblock data analysis literature, all blocks
\(\mathbf X_j ,j = 1,\ldots,J\) are assumed to be connected, and many
criteria were proposed to find block components satisfying some
covariance or correlation-based optimality. Most of them are special
cases of the optimization problem (\ref{optim_RGCCA_sample2}). These
multiblock component methods are listed in Table
\ref{multiblock_methods}. PLS path modeling is also mentioned in this
table. The great flexibility of PLS path modeling lies in the
possibility of taking into account certain hypotheses on connections
between blocks: the researcher decides which blocks are connected and
which are not.

\begin{longtable}[]{@{}
  >{\raggedright\arraybackslash}p{(\columnwidth - 6\tabcolsep) * \real{0.3304}}
  >{\centering\arraybackslash}p{(\columnwidth - 6\tabcolsep) * \real{0.0982}}
  >{\raggedright\arraybackslash}p{(\columnwidth - 6\tabcolsep) * \real{0.2857}}
  >{\centering\arraybackslash}p{(\columnwidth - 6\tabcolsep) * \real{0.2857}}@{}}
\caption{Multiblock component methods as special cases of
RGCCA.\label{multiblock_methods}}\tabularnewline
\toprule\noalign{}
\begin{minipage}[b]{\linewidth}\raggedright
\textbf{Methods}
\end{minipage} & \begin{minipage}[b]{\linewidth}\centering
\(g(x)\)
\end{minipage} & \begin{minipage}[b]{\linewidth}\raggedright
\(\tau_j\)
\end{minipage} & \begin{minipage}[b]{\linewidth}\centering
\(\mathbf{C}\)
\end{minipage} \\
\midrule\noalign{}
\endfirsthead
\toprule\noalign{}
\begin{minipage}[b]{\linewidth}\raggedright
\textbf{Methods}
\end{minipage} & \begin{minipage}[b]{\linewidth}\centering
\(g(x)\)
\end{minipage} & \begin{minipage}[b]{\linewidth}\raggedright
\(\tau_j\)
\end{minipage} & \begin{minipage}[b]{\linewidth}\centering
\(\mathbf{C}\)
\end{minipage} \\
\midrule\noalign{}
\endhead
\bottomrule\noalign{}
\endlastfoot
\textbf{SUMCOR} \citep{Horst1961} & \(x\) &
\(\tau_j = 0, j=1, \ldots, J\) &
\(\mathbf{C}_2 = \begin{pmatrix} 1 & 1 & \cdots & 1 \\ 1 & 1 & \ddots & \vdots \\ \vdots & \ddots& \ddots & 1\\ 1 & \cdots & 1 & 1 \end{pmatrix}\) \\
\textbf{SSQCOR} \citep{Kettenring1971} & \(x^2\) &
\(\tau_j = 0, j=1, \ldots, J\) & \(\mathbf{C}_2\) \\
\textbf{SABSCOR} \citep{Hanafi2007} & \(|x|\) &
\(\tau_j = 0, j=1, \ldots, J\) & \(\mathbf{C}_2\) \\
\textbf{SUMCOV-1} \citep{VandeGeer1984} & \(x\) &
\(\tau_j = 1, j=1, \ldots, J\) & \(\mathbf{C}_2\) \\
\textbf{SSQCOV-1} \citep{Hanafi2006} & \(x^2\) &
\(\tau_j = 1, j=1, \ldots, J\) & \(\mathbf{C}_2\) \\
\textbf{SABSCOV-1} \citep{Tenenhaus2011, Kramer2007} & \(|x|\) &
\(\tau_j = 1, j=1, \ldots, J\) & \(\mathbf{C}_2\) \\
\textbf{SUMCOV-2} \citep{VandeGeer1984} & \(x\) &
\(\tau_j = 1, j=1, \ldots, J\) &
\(\mathbf{C}_3 = \begin{pmatrix} 0 & 1 & \cdots & 1 \\ 1 & 0 & \ddots & \vdots\\ \vdots & \ddots& \ddots& 1\\ 1 & \cdots & 1 & 0 \end{pmatrix}\) \\
\textbf{SSQCOV-2} \citep{Hanafi2006} & \(x^2\) &
\(\tau_j = 1, j=1, \ldots, J\) & \(\mathbf{C}_3\) \\
\textbf{PLS path modeling - mode B} \citep{Wold1982, Tenenhaus2005} &
\(|x|\) & \(\tau_j = 0, j=1, \ldots, J\) & \(c_{jk}=1\) for two
connected block and \(c_{jk} = 0\) otherwise \\
\end{longtable}

Many multiblock component methods aim to simultaneously find block
components and a global component. For that purpose, we consider \(J\)
blocks, \(\mathbf X_1, \ldots,  \mathbf X_J\) connected to a
\((J + 1)\)th block defined as the concatenation of the blocks,
\(\mathbf X_{J+1} = [ \mathbf X_1 ,  \mathbf X_2, \ldots,  \mathbf X_J ]\).
Several criteria were introduced in the literature, and many are listed
below.

\begin{longtable}[]{@{}
  >{\raggedright\arraybackslash}p{(\columnwidth - 6\tabcolsep) * \real{0.3304}}
  >{\centering\arraybackslash}p{(\columnwidth - 6\tabcolsep) * \real{0.0982}}
  >{\raggedright\arraybackslash}p{(\columnwidth - 6\tabcolsep) * \real{0.2857}}
  >{\centering\arraybackslash}p{(\columnwidth - 6\tabcolsep) * \real{0.2857}}@{}}
\caption{Multiblock component methods in a situation of \(J\) blocks:
\(\mathbf X_1, \ldots,  \mathbf X_J\), connected to a \((J + 1)\)th
block defined as the concatenation of the blocks:
\(\mathbf X_{J+1} = [ \mathbf X_1 ,  \mathbf X_2, \ldots,  \mathbf X_J]\).\label{multiblock_hierarchical}}\tabularnewline
\toprule\noalign{}
\begin{minipage}[b]{\linewidth}\raggedright
\textbf{Methods}
\end{minipage} & \begin{minipage}[b]{\linewidth}\centering
\(g(x)\)
\end{minipage} & \begin{minipage}[b]{\linewidth}\raggedright
\(\tau_j\)
\end{minipage} & \begin{minipage}[b]{\linewidth}\centering
\(\mathbf{C}\)
\end{minipage} \\
\midrule\noalign{}
\endfirsthead
\toprule\noalign{}
\begin{minipage}[b]{\linewidth}\raggedright
\textbf{Methods}
\end{minipage} & \begin{minipage}[b]{\linewidth}\centering
\(g(x)\)
\end{minipage} & \begin{minipage}[b]{\linewidth}\raggedright
\(\tau_j\)
\end{minipage} & \begin{minipage}[b]{\linewidth}\centering
\(\mathbf{C}\)
\end{minipage} \\
\midrule\noalign{}
\endhead
\bottomrule\noalign{}
\endlastfoot
\textbf{Generalized CCA} \citep{Carroll1968a} & \(x^2\) &
\(\tau_j = 0, j=1, \ldots, J+1\) &
\(\mathbf{C}_4 = \begin{pmatrix} 0 & \cdots & 0 & 1 \\ \vdots & \ddots & \vdots & \vdots\\ 0 & \cdots & 0 & 1\\ 1 & \cdots & 1 & 0 \end{pmatrix}\) \\
\textbf{Generalized CCA} \citep{Carroll1968b} & \(x^2\) &
\(\tau_j=0, j=1, \ldots, J_1\) ; \(\tau_j = 1, j=J_1+1, \ldots, J\) &
\(\mathbf{C}_4\) \\
\textbf{Hierarchical PCA} \citep{Wold1996} & \(x^4\) &
\(\tau_j = 1, j=1, \ldots, J\) ; \(\tau_{J+1} = 0\) &
\(\mathbf{C}_4\) \\
\textbf{Multiple co-inertia analysis}
\citep{Chessel1996, Westerhuis1998, Smilde2003} & \(x^2\) &
\(\tau_j = 1, j=1, \ldots, J\) ; \(\tau_{J+1} = 0\) &
\(\mathbf{C}_4\) \\
\textbf{Multiple factor analysis} \citep{Escofier1994} & \(x^2\) &
\(\tau_j = 1, j=1, \ldots, J+1\) & \(\mathbf{C}_4\) \\
\end{longtable}

It is quite remarkable that the single optimization problem
(\ref{optim_RGCCA_sample2}) offers a framework for all the multiblock
component methods referenced in Tables
\ref{twoblock_methods}-\ref{multiblock_hierarchical}. It has immediate
practical consequences for a unified statistical analysis and
implementation strategy. In the next section, we present the
straightforward gradient-based Algorithm for solving the RGCCA
optimization problem. This Algorithm is monotonically convergent and
hits a stationary point at convergence. Two numerically equivalent
approaches for solving the RGCCA optimization problem are available. A
primal formulation described in \cite{Tenenhaus2011, Tenenhaus2017}
requires handling matrices of dimensions \(p_j \times p_j\). A dual
formulation described in \cite{Tenenhaus2015} requires handling matrices
of dimension \(n \times n\). Therefore, the primal formulation of the
RGCCA algorithm will be preferred when \(n>p_j\) and the dual form will
be used when \(n \le p_j\). The \code{rgcca()} function of the
\pkg{RGCCA} package implements these two formulations and automatically
selects the best one.

\subsection{Sparse generalized canonical correlation
analysis}\label{sparse-generalized-canonical-correlation-analysis}

RGCCA is a component-based approach that aims to study the relationships
between several sets of variables. The quality and interpretability of
the RGCCA block components
\(\mathbf{y}_j= \mathbf{X}_j \mathbf{a}_j,j=1, \ldots,J\) are likely to
be affected by the usefulness and relevance of the variables in each
block. Therefore, it is important to identify within each block which
subsets of significant variables are active in the relationships between
blocks. For instance, biomedical data are known to be measurements of
intrinsically parsimonious processes. Sparse generalized canonical
correlation analysis (SGCCA) extends RGCCA to address this issue of
variable selection \citep{Tenenhaus2014b} and enhances the RGCCA
framework. The SGCCA optimization problem is obtained by considering
another set \(\Omega_j\) as follows:

\begin{equation}
\displaystyle \underset{\mathbf{a}_1, \ldots,\mathbf{a}_J}{\text{maximize }} \sum_{j, k = 1}^J c_{jk}\text{g}\left(\widehat{\text{cov}}\left(\mathbf{X}_j\mathbf{a}_j, \mathbf{X}_k\mathbf{a}_k\right)\right) \text{ s.t. } \Vert \mathbf{a}_j \Vert_2 \le 1 \text{ and } \Vert \mathbf{a}_j \Vert_1 \le s_j, j=1,\ldots,J.
\label{optim_SGCCA}
\end{equation}

\(s_j\) is a user-defined positive constant that determines the amount
of sparsity for \(\mathbf{a}_j, j=1, \ldots,J\). The smaller the
\(s_j\), the larger the degree of sparsity for \(\mathbf{a}_j\). The
sparsity parameter \(s_j\) is usually set by cross-validation or
permutation procedures. Alternatively, values of \(s_j\) can be chosen
to result in desired amounts of sparsity.

SGCCA is also implemented in the \pkg{RGCCA} package and offers a sparse
counterpart for all the covariance-based methods cited above.

\subsection{A master algorithm for the RGCCA
framework}\label{a-master-algorithm-for-the-rgcca-framework}

The general optimization problem behind the RGCCA framework can be
formulated as follows:

\begin{align}
\underset{ \mathbf a_1, \ldots, \mathbf a_J}{\text{max}} f( \mathbf a_1, \ldots, \mathbf a_J)
\text{ s.t. }  \mathbf a_j \in \Omega_j, \text{ } j = 1, \ldots, J,
\label{master_optim}
\end{align} where
\(f( \mathbf a_1, \ldots, \mathbf a_J):\mathbb{R}^{p_1}\times \ldots \times \mathbb{R}^{p_J} \xrightarrow{}\mathbb{R}\)
is a continuously differentiable multi-convex function and each
\(\mathbf a_j\) belongs to a compact set
\(\Omega_j \subset \mathbb{R}^{p_j}\). For such a function defined over
a set of parameter vectors \(( \mathbf a_1, \ldots, \mathbf a_J)\), we
make no difference between the notations
\(f( \mathbf a_1, \ldots, \mathbf a_J)\) and \(f( \mathbf a)\), where
\(\mathbf a\) is the column vector
\(\mathbf a = \left(  \mathbf a_1^\top, \ldots,  \mathbf a_J^\top\right)^\top\)
of size \(p = \sum_{j=1}^{J}p_j\). Moreover, the vertical concatenation
of a column vector is denoted
\(\mathbf a = \left(  \mathbf a_1; \ldots;  \mathbf a_J \right)\) for
the sake of simplification of notation.

A simple, monotonically, and globally convergent algorithm is presented
for solving the optimization problem (\ref{master_optim}). The
maximization of the function \(f\) defined over different parameter
vectors (\(\mathbf a_1, \ldots, \mathbf a_J\)) is approached by updating
each of parameter vector in turn, keeping the others fixed. This update
rule was recommended in \cite{DeLeeuw1994} and is called Block
Relaxation or cyclic Block Coordinate Ascent (BCA).

Let \(\nabla_j f( \mathbf a)\) be the partial gradient of
\(f( \mathbf a)\) with respect to \(\mathbf a_j\). We assume
\(\nabla_j f( \mathbf a) \neq \mathbf{0}\) in this manuscript. This
assumption is not too binding as \(\nabla_j f( \mathbf a) = \mathbf{0}\)
characterizes the global minimum of
\(f( \mathbf a_1 , \ldots ,  \mathbf a_J )\) with respect to
\(\mathbf a_j\) when the other vectors
\(\mathbf a_1 , \ldots ,  \mathbf a_{j-1} ,  \mathbf a_{j+1} , \ldots ,  \mathbf a_J\)
are fixed.

We want to find an update \(\hat{ \mathbf a}_j\in \Omega_j\) such that
\(f( \mathbf a)\leq f( \mathbf a_1, ...,  \mathbf a_{j-1}, \hat{ \mathbf a}_j,  \mathbf a_{j+1}, ...,  \mathbf a_J)\).
As \(f\) is a continuously differentiable multi-convex function and
considering that a convex function lies above its linear approximation
at \(\mathbf a_j\) for any \(\tilde{ \mathbf a}_j\in\Omega_j\), the
following inequality holds:

\begin{equation}
\begin{gathered}
f( \mathbf a_1, ...,  \mathbf a_{j-1}, \tilde{ \mathbf a}_j,  \mathbf a_{j+1}, \ldots,  \mathbf a_J) \geq f( \mathbf a) + \nabla_jf( \mathbf a)^\top(\tilde{ \mathbf a}_j -  \mathbf a_j).
\label{minorizing_ineq}
\end{gathered}
\end{equation}

On the right-hand side of the inequality (\ref{minorizing_ineq}), only
the term \(\nabla_jf( \mathbf a)^\top\tilde{ \mathbf a}_j\) is relevant
to \(\tilde{ \mathbf a}_j\) and the solution that maximizes the
minorizing function over \(\tilde{ \mathbf a}_j\in\Omega_j\) is obtained
by considering the following optimization problem:

\begin{equation}
\hat{ \mathbf a}_j = \underset{\tilde{ \mathbf a}_j\in\Omega_j}{\text{argmax }} \nabla_j f( \mathbf a)^\top \tilde{ \mathbf a}_j := r_j( \mathbf a).
\label{core_update}
\end{equation}

The entire algorithm is subsumed in Algorithm \ref{master_algo}.

\begin{algorithm}[!ht]
    \caption{Algorithm for the maximization of a continuously differentiable multi-convex function}
    \begin{algorithmic}[1]
        \STATE {\bfseries Result:} {$\mathbf a_1^s, \ldots,  \mathbf a_J^s$ (approximate solution of (\ref{master_optim}))}
        \STATE {\bfseries Initialization:} {choose random vector $\mathbf a_j^0\in\Omega_j, j =1, \ldots, J$, $\varepsilon$;}
        \STATE$s = 0$ ;
        \REPEAT
        \FOR{$j=1$ {\bfseries to} $J$}
        \STATE \hspace{-2cm}$\vcenter{\begin{equation}
             \mathbf a_j^{s+1} = r_j\left(  \mathbf a_1^{s+1}, \ldots,  \mathbf a_{j-1}^{s+1},  \mathbf a_j^{s}, \ldots,  \mathbf a_J^{s}\right).
        \end{equation}}$
        \ENDFOR
        \STATE$s = s + 1$ ;
        \UNTIL{$f( \mathbf a_1^{s+1}, \ldots,  \mathbf a_J^{s+1})-f( \mathbf a_1^s, \ldots,  \mathbf a_J^s) < \varepsilon$}
    \end{algorithmic}
    \label{master_algo}
\end{algorithm}

We need to introduce some extra notations to present the convergence
properties of Algorithm \ref{master_algo}:
\(\Omega = \Omega_1 \times \ldots \times \Omega_J\),
\(\mathbf a = \left( \mathbf a_1; \ldots; \mathbf a_J\right) \in \Omega\),
\(c_j \text{ : } \Omega\mapsto\Omega\) is an operator defined as
\(c_j( \mathbf a) = \left( \mathbf a_1; \ldots;  \mathbf a_{j-1} ; r_j( \mathbf a) ;  \mathbf a_{j+1} ; \ldots;  \mathbf a_J\right)\)
with \(r_j( \mathbf a)\) introduced in
Equation\textasciitilde{}\ref{core_update} and
\(c \text{ : } \Omega\mapsto\Omega\) is defined as
\(c = c_J\circ c_{J-1}\circ ... \circ c_1\), where \(\circ\) stands for
the composition operator. Using the operator \(c\), the
\guillemotleft for loop\guillemotright{} inside Algorithm
\ref{master_algo} can be replaced by the following recurrence relation:
\(\mathbf a^{s+1} = c( \mathbf a^s)\). The convergence properties of
Algorithm \ref{master_algo} are summarized in the following proposition:

\begin{proposition}
    Let $\left\lbrace  \mathbf a^s\right\rbrace_{s=0}^{\infty}$ be any sequence generated by the recurrence relation $\mathbf a^{s+1} = c( \mathbf a^s)$ with $\mathbf a^0\in\Omega$. Then, the following properties hold:
    \begin{enumerate}[topsep=0pt,itemsep=-0.75ex,partopsep=1ex,parsep=1ex, label = {(\alph*)}]
        \item  \label{prop_pt1} The sequence $\left\lbrace f( \mathbf a^s)\right\rbrace $ is monotonically increasing and therefore convergent as $f$ is bounded on $\Omega$. This result implies the monotonic convergence of Algorithm \ref{master_algo}.
        \item  \label{prop_pt2} If the infinite sequence $\left\lbrace f( \mathbf a^s)\right\rbrace $ involves a finite number of distinct terms, then the last distinct point satisfies $c( \mathbf a^s) =  \mathbf a^s$ and therefore is a stationary point of the problem (\ref{master_optim}). 
    \item  \label{prop_pt3} $\underset{s\xrightarrow[]{}\infty}\lim{f( \mathbf a^s) = f( \mathbf a)}$, where $\mathbf a$ is a fixed point of $c$.
        \item  \label{prop_pt4} The limit of any convergent subsequence of $\left\lbrace  \mathbf a^s\right\rbrace $ is a fixed point of $c$.
        \item  \label{prop_pt5} The sequence $\left\lbrace  \mathbf a^s \right\rbrace $ is asymptotically regular: $\underset{s\xrightarrow[]{}\infty}\lim{\sum_{j=1}^{J} \Vert  \mathbf a_j^{s+1} -  \mathbf a_j^s \Vert} = 0$. This result implies that if the threshold $\varepsilon$ for the stopping criterion in Algorithm \ref{master_algo} is made sufficiently small, the output of Algorithm \ref{master_algo} will be as close as wanted to a stationary point of (\ref{master_optim}). 
        \item  \label{prop_pt6} If the equation $\mathbf a = c( \mathbf a)$ has a finite number of solutions, then the sequence $\left\lbrace  \mathbf a^s\right\rbrace $ converges to one of them.
    \end{enumerate}
    \label{cv_prop}
\end{proposition}

Proposition \ref{cv_prop} gathers all the convergence properties of
Algorithm \ref{master_algo}. The three first points of Proposition
\ref{cv_prop} concern the behavior of the sequence values
\(\left\lbrace f( \mathbf a^s) \right\rbrace\) of the objective
function, whereas the three last points are about the behavior of the
sequence \(\left\lbrace  \mathbf a^s \right\rbrace\). The full proof of
these properties is given in \cite{Tenenhaus2017}.

The optimization problem \eqref{optim_RGCCA} defining the RGCCA
framework is a particular case of \eqref{master_optim}. Indeed, we
assume the \(\Omega_j\)s' to be compact. In addition, when the diagonal
of \(\mathbf{C}\) is null, the convexity and the continuous
differentiability of the function g imply that the objective function of
\eqref{optim_RGCCA} itself is multi-convex continuously differentiable.
When at least one element of the diagonal of \(\mathbf{C}\) is different
from \(0\), additional conditions have to be imposed on g to keep the
objective function multi-convex. For example, when g is twice
differentiable, a sufficient condition is that
\(\forall x\in\mathbb{R}_+, \text{ } g'(x)\geq 0\). This condition
guarantees that the second derivative of
\(g\left(\mathbf{a}_j^\top \mathbf{\Sigma}_{jj} \mathbf{a}_j\right)\) is
positive-definite:

\begin{equation}
\frac{\partial^2 g\left(\mathbf a_j^\top  \mathbf \Sigma_{jj} \mathbf a_j \right)}{\partial \mathbf a_j \partial \mathbf a_j^\top} = 2 \left[ g'\left( \mathbf a_j^\top  \mathbf \Sigma_{jj} \mathbf a_j \right) \mathbf \Sigma_{jj} + 2 g''\left(\mathbf a_j^\top  \mathbf \Sigma_{jj} \mathbf a_j\right)  \mathbf \Sigma_{jj}\mathbf a_j\mathbf a_j^\top \mathbf \Sigma_{jj} \right]. 
\end{equation}

All functions g considered in the RGCCA framework satisfy this
condition. Consequently, the optimization problem (\ref{optim_RGCCA})
falls under the umbrella of the general optimization framework presented
in the previous section.

\subsection{The specific case of
RGCCA}\label{the-specific-case-of-rgcca}

The optimization problem \eqref{optim_RGCCA_sample1} defining
sample-based RGCCA is a particular case of \eqref{master_optim}. Indeed,
\(\Omega_j = \{ \mathbf a_j \in \mathbb{R}^{p_j}; \mathbf a_j^\top \widehat{\mathbf{\Sigma}}_{jj} \mathbf a_j = 1\}\)
defines a compact set. Therefore, Algorithm \ref{master_algo} can be
used to solve the optimization problem (\ref{optim_RGCCA_sample1}). This
is done by updating each parameter vector, in turn, keeping the others
fixed. Hence, we want to find an update
\(\hat{\mathbf{a}}_j\in \Omega_j=\left\lbrace \mathbf a_j \in \mathbb{R}^{p_j}; \mathbf{a}_j^\top \widehat{\mathbf{\Sigma}}_{jj} \mathbf{a}_j = 1 \right\rbrace\)
such that
\(f(\mathbf{a})\leq f(\mathbf{a}_1, \ldots, \mathbf{a}_{j-1}, \hat{\mathbf{a}}_j, \mathbf{a}_{j+1}, \ldots, \mathbf{a}_J)\).
The RGCCA update is obtained by considering the following optimization
problem:

\begin{equation}
\hat{\mathbf{a}}_j = \underset{\tilde{\mathbf{a}}_j\in\Omega_j}{\text{argmax }}  \nabla_j f(\mathbf{a})^\top \tilde{\mathbf{a}}_j = \frac{ \widehat{\mathbf{\Sigma}}_{jj}^{-1} \mathbf \nabla_j f( \mathbf a)}{\Vert  \widehat{\mathbf{\Sigma}}_{jj}^{-1/2} \mathbf \nabla_j f( \mathbf a) \Vert} := r_j( \mathbf a), j=1, \ldots, J,
\label{RGCCA_update}
\end{equation} where the partial gradient
\(\mathbf \nabla_j f( \mathbf a)\) of \(f( \mathbf a)\) with respect to
\(\mathbf a_j\) is a \(p_j\)-dimensional column vector given by:

\begin{equation}
 \mathbf \nabla_j f( \mathbf a)=2\sum_{k=1}^{J}c_{jk}g'\left( \mathbf a_j^\top  \widehat{\mathbf{\Sigma}}_{jk}  \mathbf a_k \right)  \widehat{\mathbf{\Sigma}}_{jk}  \mathbf a_k.
\label{grad_obj_function}
\end{equation}

\subsection{The specific case of
SGCCA}\label{the-specific-case-of-sgcca}

The optimization problem (\ref{optim_SGCCA}) falls into the RGCCA
framework with
\(\Omega_j = \lbrace  \mathbf a_j\in\mathbb{R}^{p_j}; \Vert  \mathbf a_j \Vert_2 \leq 1; \Vert  \mathbf a_j \Vert_1 \leq s_l\rbrace\).
\(\Omega_j\) is defined as the intersection between the \(\ell_2\)-ball
of radius \(1\) and the \(\ell_1\)-ball of radius
\(s_l \in \mathbb{R}_+^\star\) which are two compact sets. Hence,
\(\Omega_j\) is a compact set. Therefore, we can consider the following
update for SGCCA:

\begin{equation}
    \hat{ \mathbf a}_j = \underset{\Vert \tilde{ \mathbf a}_j \Vert_2 \leq 1 ; \Vert \tilde{ \mathbf a}_j \Vert_1 \leq s_j}{\text{argmax }} \nabla_j f( \mathbf a)^\top \tilde{ \mathbf a}_j := r_j( \mathbf a).
\label{update_SGCCA}
\end{equation}

According to \cite{Witten2009a}, the solution of (\ref{update_SGCCA})
satisfies:

\begin{equation}
    r_j( \mathbf a) = \hat{ \mathbf a}_j = \frac{\mathcal{S}(\nabla_j f( \mathbf a), \lambda_j)}{\Vert \mathcal{S}(\nabla_j f( \mathbf a), \lambda_j)\Vert_2}, \text{ where } \lambda_j = \left\lbrace\begin{array}{ccc}
    0 \text{ if } & \frac{\Vert \nabla_j f( \mathbf a) \Vert_1}{\Vert \nabla_j f( \mathbf a) \Vert_2} & \leq s_j\\
    \text{find } \lambda_j \text{ such that } & \Vert \hat{ \mathbf a}_j \Vert_1 & = s_j    \end{array}\right.,
    \label{SGCCA_sol}
\end{equation}

where function \(\mathcal{S}(., \lambda)\) is the soft-thresholding
operator. When applied on a vector \(\mathbf x\in\mathbb{R}^p\), this
operator is defined as:

\begin{equation}
     \mathbf u = \mathcal{S}(\mathbf x, \lambda) \Leftrightarrow u_j = \left\lbrace
    \begin{array}{ccc}
        {\mathrm{sign}}(x_j)(|x_j| -  \lambda), & \text{ if } |x_j| &> \lambda\\
        0, & \text{ if } |x_j| &\leq \lambda\\ 
    \end{array}\right., j = 1, \ldots, p.
\end{equation}

We made the assumption that the \(\ell_2\)-ball of radius \(1\) is not
included in the \(\ell_1\)-ball of radius \(s_j\) and the other way
round. Otherwise, systematically, only one of the two constraints is
active. This assumption is true when the corresponding spheres
intersect. This assumption can be translated into conditions on \(s_j\).

The norm equivalence between \(\Vert . \Vert_1\) and \(\Vert . \Vert_2\)
can be formulated as the following inequality:

\begin{equation}
    \forall \mathbf x \in \mathbb{R}^{p_j}, \text{ } \Vert \mathbf x \Vert_2 \leq \Vert \mathbf x \Vert_1 \leq \sqrt{p_j}\Vert \mathbf x \Vert_2.
\label{existence_conditions}
\end{equation}

This can be converted into a condition on \(s_j\):
\(1 \leq s_j \leq \sqrt{p_j}\). When such a condition is fulfilled, the
\(\ell_2\)-sphere of radius \(1\) and the \(\ell_1\)-sphere of radius
\(s_j\) necessarily intersect. Within the \pkg{RGCCA} package, for
consistency with the value of \(\tau_j \in [0, 1]\), the level of
sparsity for \(\mathbf a_j\) is controlled with
\(s_j/p_j \in [1/\sqrt{p_j}, 1]\).

Several strategies, such as binary search or the projection on convex
sets algorithm (POCS), also known as the alternating projection method
\citep{Boyd2003}, can be used to determine the optimal \(\lambda_j\)
verifying the \(\ell_1\)-norm constraint. Here, a much faster approach
described in \cite{Gloaguen2017} is implemented within the \pkg{RGCCA}
package.

The SGCCA algorithm is similar to the RGCCA algorithm and keeps the same
convergence properties. Empirically, we note that the S/RGCCA algorithm
is found to be not sensitive to the starting point and usually reaches
convergence (\code{tol = 1e-16}) within a few iterations.

\subsection{Higher level RGCCA
algorithm}\label{higher-level-rgcca-algorithm}

In many applications, several components per block need to be
identified. The traditional approach incorporates the single-unit RGCCA
algorithm in a deflation scheme and sequentially computes the desired
number of components. More precisely, the RGCCA optimization problem
returns a set of \(J\) optimal block-weight vectors, denoted here
\(\mathbf a_j^{(1)}, \text{ } j = 1, \ldots, J\). Let
\(\mathbf y_j^{(1)} = \mathbf X_j \mathbf a_j^{(1)}, \text{ } j = 1, \ldots, J\)
be the corresponding block components. Two strategies to determine
higher-level weight vectors are presented: the first yields orthogonal
block components, and the second yields orthogonal weight vectors.
Deflation is the most straightforward way to add orthogonality
constraints. This deflation procedure is sequential and consists in
replacing within the RGCCA optimization problem the data matrix
\(\mathbf X_j\) by \(\mathbf X_j^{(1)}\) its projection onto either: (i)
the orthogonal subspace of \(\mathbf y_j^{(1)}\) if orthogonal
components are desired:
\(\mathbf X_j^{(1)} = \mathbf X_j -  \mathbf y_j^{(1)} \left( { \mathbf y_j^{(1)}}^\top  \mathbf y_j^{(1)} \right)^{-1}{ \mathbf y_j^{(1)}}^\top \mathbf X_j\),
or (ii) the orthogonal subspace of \(\mathbf a_j^{(1)}\) for orthogonal
weight vectors
\(\mathbf X_j^{(1)} = \mathbf X_j - \mathbf X_j \mathbf a_j^{(1)} \left( { \mathbf a_j^{(1)}}^\top  \mathbf a_j^{(1)} \right)^{-1}{ \mathbf a_j^{(1)}}^\top\).

The second level RGCCA optimization problem boils down to:

\begin{equation}
        \underset{ \mathbf a_1, \ldots, \mathbf a_J}{\text{max }} \sum_{j, k = 1}^J c_{jk} \text{ g}\left(n^{-1} \mathbf a_j^\top {\mathbf X_j^{(1)}}^\top \mathbf X_k^{(1)}  \mathbf a_k \right)
        \text{ s.t. }  \mathbf a_j \in \Omega_j.
    \label{optim_RGCCA_orth_comp}
\end{equation}

The optimization problem (\ref{optim_RGCCA_orth_comp}) is solved using
Algorithm \ref{master_algo} and returns a set of optimal block-weight
vectors \(\mathbf a_j^{(2)}\) and block components
\(\mathbf y_j^{(2)} = \mathbf X_j^{(1)} \mathbf a_j^{(2)}\), for
\(j = 1\ldots, J\).

For orthogonal block weight vectors,
\(\mathbf y_j^{(2)} = \mathbf X_j^{(1)} \mathbf a_j^{(2)} = \mathbf X_j  \mathbf a_j^{(2)}\)
naturally expresses as a linear combination of the original variables.
For orthogonal block component, as
\(\mathbf y_j^{(1)} = \mathbf X_j \mathbf a_j^{(1)}\), the range space
of \(\mathbf X_j^{(1)}\) is included in the range space of
\(\mathbf X_j\). Therefore any block component \(\mathbf y_j^{(2)}\)
belonging to the range space of \(\mathbf X_j^{(1)}\) can also be
expressed in terms of the original block \(\mathbf X_j\): that is, it
exists \(\mathbf a_j^{(2)^\star}\) such that
\(\mathbf y_j^{(2)} = \mathbf X_j^{(1)} \mathbf a_j^{(2)} = \mathbf X_j \mathbf a_j^{(2)^\star}\).
It implies that the block components can always be expressed in terms of
the original data matrix, whatever the deflation mode.

This deflation procedure can be iterated in a very flexible way. For
instance, it is not necessary to keep all the blocks in the procedure at
all stages: the number of components per block can vary from one block
to another. This might be interesting in a supervised setting where we
want to predict a univariate block from other blocks. In that case, the
deflation procedure applies to all blocks except the one to predict.

To conclude this section, when the superblock option is used, various
deflation strategies (what to deflate and how) have been proposed in the
literature. We propose, as the default option, to deflate only the
superblock with respect to its global components:

\[\mathbf X_{J+1}^{(1)} = \left(\mathbf{I} -  \mathbf y_{J+1}^{(1)} \left( { \mathbf y_{J+1}^{(1)}}^\top  \mathbf y_{J+1}^{(1)} \right)^{-1}{ \mathbf y_{J+1}^{(1)}}^\top \right) \mathbf X_{J+1} = \left[ \mathbf X_1^{(1)}, \ldots, \mathbf X_J^{(1)} \right].\]

The individual blocks \(\mathbf X_j^{(1)}\)s are then retrieved from the
deflated superblock. This strategy enables recovering multiple factor
analysis (\code{ade4::mfa()} and \code{FactoMineR::MFA()}). Note that,
in this case, block components do not belong to their block space and
are correlated. On the contrary, we follow the deflation strategy
described in \cite{Chessel1996} (\code{ade4::mcoa()}) for multiple
co-inertia analysis, which is one of the most popular and established
methods of the multiblock literature.

\subsection{Average variance
explained}\label{average-variance-explained}

In this section, using the idea of average variance explained (AVE), the
following indicators of model quality are defined:

\begin{itemize}
\tightlist
\item
  The AVE for a given block component \(\mathbf{y}_j\) can be computed
  using the following formula:
\end{itemize}

\begin{equation}
\mathrm{AVE}(\mathbf X_j)=  \frac{1}{\Vert \mathbf X_j \Vert^2} \sum_{h=1}^{p_j} \text{var}(\mathbf{x}_{jh}) \times \text{cor}^2(\mathbf{x}_{jh},\mathbf{y}_j).
\label{AVE_X}
\end{equation}

\begin{itemize}
\tightlist
\item
  A global indicator of model quality can be obtained by considering a
  weighted sum of these individual AVEs. This outer AVE is defined as:
\end{itemize}

\begin{equation}
\displaystyle \mathrm{AVE(outer model)} = \left( 1/\sum_j \Vert \mathbf X_j \Vert^2 \right) \sum_j \Vert \mathbf X_j \Vert^2 \mathrm{AVE}(\mathbf X_j).
\label{AVE_outer}
\end{equation}

However, the previous quantities do not take into account the
correlations between blocks. Therefore, another indicator of model
quality is the inner AVE, defined as follows:

\begin{equation}
\displaystyle \mathrm{AVE(inner model)} = \left( 1/\sum_{j<k} c_{jk} \right) \sum_{j<k} c_{jk} \mathrm{cor}^2(\mathbf{y}_j , \mathbf{y}_k).
\label{AVE_inner}
\end{equation}

All these quantities vary between 0 and 1 and reflect important
properties of the model.

Equation\textasciitilde{}\ref{AVE_X} is defined for a specific block
component. The cumulative AVE is obtained by summing these individual
AVEs over the different components. However, this sum applies only to
orthogonal block components. For correlated components, we follow the
QR-orthogonalization procedure described in \cite{Zou2006} to consider
only the increase of AVE due to adding the new components.

Guidelines describing R/SGCCA in practice are provided in
\cite{Garali2018}. The usefulness and versatility of the \pkg{RGCCA}
package are illustrated in the next section.

\section{The RGCCA package}\label{the-rgcca-package}

This section describes how the \pkg{RGCCA} package is designed and how
it can be used on two real datasets.

\subsection{Software design}\label{software-design}

The main user-accessible functions are listed in Table
\ref{exported_functions}.

\begin{longtable}[]{@{}
  >{\raggedright\arraybackslash}p{(\columnwidth - 2\tabcolsep) * \real{0.4386}}
  >{\raggedright\arraybackslash}p{(\columnwidth - 2\tabcolsep) * \real{0.5614}}@{}}
\caption{Functions implemented in the \pkg{RGCCA} package.
\label{exported_functions}}\tabularnewline
\toprule\noalign{}
\begin{minipage}[b]{\linewidth}\raggedright
Function
\end{minipage} & \begin{minipage}[b]{\linewidth}\raggedright
Description
\end{minipage} \\
\midrule\noalign{}
\endfirsthead
\toprule\noalign{}
\begin{minipage}[b]{\linewidth}\raggedright
Function
\end{minipage} & \begin{minipage}[b]{\linewidth}\raggedright
Description
\end{minipage} \\
\midrule\noalign{}
\endhead
\bottomrule\noalign{}
\endlastfoot
\code{rgcca} & Main entry point of the package, this function allows
fitting a R/SGCCA model on a multiblock dataset. \\
\code{rgcca\_cv} & Find the best set of parameters for a R/SGCCA model
using cross-validation. \\
\code{rgcca\_predict} & Train a caret model on the block components of a
fitted R/SGCCA model and predict values for unseen individuals. \\
\code{rgcca\_transform} & Use a fitted R/SGCCA model to compute the
block components of unseen individuals. \\
\code{rgcca\_permutation} & Find the best set of parameters for a
R/SGCCA model using a permutation strategy. \\
\code{rgcca\_bootstrap} & Evaluate the significance of the block-weight
vectors produced by a R/SGCCA model using bootstrap. \\
\code{rgcca\_stability} & Select the most stable variables of a R/SGCCA
model using their VIPs. \\
\code{summary/print/plot} & Summary, print and plot methods for outputs
of functions \code{rgcca}, \code{rgcca\_cv}, \code{rgcca\_permutation},
\code{rgcca\_bootstrap}, and \code{rgcca\_stability}. \\
\end{longtable}

These functions are detailed hereafter.

\begin{itemize}
\item
  \code{rgcca()}
\end{itemize}

The cornerstone of the RGCCA package is the \code{rgcca()} function.
This function enables the construction of a flexible pipeline
encompassing all model-building steps. This master function
automatically checks the proper formatting of the blocks and performs
preprocessing steps based on the \code{scale_block} and \code{scale}
arguments. Several user-specified strategies for handling missing data
are available through the \code{NA_method} argument. The final step of
the pipeline explicitly focuses on the choice of the multiblock
component methods. The list of pre-specified multiblock component
methods that can be used within the \pkg{RGCCA} package is reported
below:

\footnotesize

\begin{CodeChunk}
\begin{CodeInput}
R> RGCCA::available_methods()
\end{CodeInput}
\begin{CodeOutput}
 [1] "rgcca"     "sgcca"     "pca"       "spca"      "pls"       "spls"     
 [7] "cca"       "ifa"       "ra"        "gcca"      "maxvar"    "maxvar-b" 
[13] "maxvar-a"  "mfa"       "mcia"      "mcoa"      "cpca-1"    "cpca-2"   
[19] "cpca-4"    "hpca"      "maxbet-b"  "maxbet"    "maxdiff-b" "maxdiff"  
[25] "sabscor"   "ssqcor"    "ssqcov-1"  "ssqcov-2"  "ssqcov"    "sumcor"   
[31] "sumcov-1"  "sumcov-2"  "sumcov"    "sabscov-1" "sabscov-2"
\end{CodeOutput}
\end{CodeChunk}

\normalsize

These method can be retrieved by setting the \code{method} argument of
the \code{rgcca()} function. Alternatively, several arguments
(\code{tau}, \code{sparsity}, \code{scheme}, \code{connection},
\code{comp_orth}, \code{superblock}) allow users to manually recover the
various methods listed in Section 2.3.

\begin{itemize}
\item
  \code{rgcca\_cv}
\end{itemize}

The optimal tuning parameters can be determined by cross-validating
different indicators of quality, namely:

\begin{enumerate}
\item For classification: \code{Accuracy}, \code{Kappa}, \code{F1}, \code{Sensitivity}, \code{Specificity}, \code{Pos\_Pred\_Value}, \code{Neg\_Pred\_Value}, \code{Precision}, \code{Recall}, \code{Detection\_Rate}, and \code{Balanced\_Accuracy}. 

\item For regression: \code{RMSE} and \code{MAE}.
\end{enumerate}

This cross-validation protocol is made available through the
\code{rgcca\_cv()} function. This function can be used in the presence
of a response block specified by the \code{response} argument. The
primary objective of this function is to automatically find the set of
tuning parameters (shrinkage parameters, amount of sparsity, number of
components) that yields the highest cross-validated scores. The
prediction model aims to predict the response block from all available
block components. \code{rgcca\_cv()} harnesses the power of the
\code{caret} package. As a direct consequence, an astonishingly large
number of prediction models becomes accessible (for an exhaustive list,
refer to \code{caret::modelLookup()}). \code{rgcca\_cv()} calls
\code{rgcca\_transform()} to obtain the block components associated with
the test set and \code{rgcca\_predict()} for making predictions.

\begin{itemize}
\item
  \code{rgcca\_permutation()}
\end{itemize}

When blocks play a symmetric role (i.e., without a specific response
block), a permutation-based strategy, very similar to the one proposed
in \cite{Witten2009a} has also been integrated within the \pkg{RGCCA}
package through the \code{rgcca\_permutation()} function.

For each set of candidate parameters, the following steps are performed:

\begin{enumerate}
\item S/RGCCA is run on the original data 
$\mathbf X_1, \ldots, \mathbf X_J$, and we record the value of the objective 
function, denoted as $t$.
\item \code{n\_perm} times, the rows of $\mathbf X_1, \ldots, \mathbf X_J$ are
randomly permuted to obtained permuted data sets $\mathbf X_1^*, \ldots, \mathbf X_J^*$.
S/RGCCA is then run on these permuted data sets, and we record the value of the objective
function, denoted as $t^*$.
\item The resulting p-value is determined by calculting the fraction of permuted $t^*$ values that exceeds the non-permuted $t$ value.
\item The resulting zstat is defined as $\frac{t-\text{mean}(t^*)}{\text{sd}(t^*)}$.
\end{enumerate}

The best set of tuning parameters is then the set that yields the
highest zstat. This procedure is available through the
\code{rgcca\_permutation()} function.

\begin{itemize}
\item
  \code{rgcca\_bootstrap()}
\end{itemize}

Once a fitted model has been obtained, the \code{rgcca\_bootstrap()}
function can be used to assess the reliability of parameter estimates
(block-weight/loading vectors). Bootstrap samples of the same size as
the original data are repeatedly sampled with replacement from the
original data. RGCCA is then applied to each bootstrap sample to obtain
the RGCCA estimates. We calculate the standard deviation of the
estimates across the bootstrap samples, from which we derive bootstrap
confidence intervals, t-ratio (defined as the ratio of the parameter
estimate to its bootstrap estimate of the standard deviation), and
p-value (the p-value is computed by assuming that the ratio of the
parameter estimate to its standard deviation follows the standardized
normal distribution), to indicate the reliability of parameter
estimates.

Since multiple p-values are constructed simultaneously, correction for
multiple testing applies and, adjusted p-values are returned.

\begin{itemize}
\item
  \code{rgcca\_stability()}
\end{itemize}

We also implemented a procedure to stabilize the selection of variables
in SGCCA. \cite{Tenenhaus1998} defines the variable importance in
projection (VIP) score for the PLS method. This score can be used for
variable selection: the higher the score, the more important the
variable. We use this idea to propose a procedure for evaluating the
stability of the variable selection procedure of SGCCA. This procedure
relies on the following score: \begin{equation}
\displaystyle \mathrm{VIP}(\mathbf{x}_{jh}) = \frac{1}{K} \sum_{k=1}^K \left(\mathbf{a}_{jh}^{(k)2} \mathrm{AVE}\left(\mathbf X_j^{(k)}\right)\right).
\label{VIP}
\end{equation} SGCCA is applied several times using a bootstrap
resampling procedure. For each fitted model, the VIPs are computed for
each variable. The higher VIPs averaged over the different models are
kept. This procedure is available through the \code{rgcca\_stability()}
function.

The outputs of most of the aforementioned functions can be described and
visualized using the generic \code{summary()}, \code{print()} and
\code{plot()} functions.

The following two sections provide examples illustrating the practical
applications of these functions.

\subsection{Case study: the Russett
dataset}\label{case-study-the-russett-dataset}

In this section, we reproduce some of the results presented in
\cite{Tenenhaus2011} from the Russett data. The Russett dataset is
available within the \pkg{RGCCA} package. The Russett dataset
\citep{Russett1964} is studied in \cite{Gifi1990}. Russett collected
this data to study relationships between agricultural inequality,
industrial development, and political instability.

\footnotesize

\begin{CodeChunk}
\begin{CodeInput}
R> library("RGCCA")
R> data("Russett")
R> colnames(Russett)
\end{CodeInput}
\begin{CodeOutput}
 [1] "gini"     "farm"     "rent"     "gnpr"     "labo"     "inst"    
 [7] "ecks"     "death"    "demostab" "demoinst" "dictator"
\end{CodeOutput}
\end{CodeChunk}

\normalsize

The first step of the analysis is to define the blocks. Three blocks of
variables have been defined for 47 countries. The variables that compose
each block have been defined according to the nature of the variables.

\begin{itemize}
\tightlist
\item
  The first block \(\mathbf{X}_1\) = {[}\code{gini}, \code{farm},
  \code{rent}{]} is related to agricultural inequality:

  \begin{itemize}
  \tightlist
  \item
    \code{gini} = Inequality of land distribution,
  \item
    \code{farm} = \% farmers that own half of the land (\textgreater{}
    50),
  \item
    \code{rent} = \% farmers that rent all their land.
  \end{itemize}
\item
  The second block \(\mathbf{X}_2\) = {[}\code{gnpr}, \code{labo}{]}
  describes industrial development:

  \begin{itemize}
  \tightlist
  \item
    \code{gnpr} = Gross national product per capita (\$1955),
  \item
    \code{labo} = `\% of labor force employed in agriculture.
  \end{itemize}
\item
  The third one \(\mathbf{X}_3\) = {[}\code{inst}, \code{ecks},
  \code{death}{]} measures political instability:

  \begin{itemize}
  \tightlist
  \item
    \code{inst} = Instability of executive (45-61),
  \item
    \code{ecks} = Number of violent internal war incidents (46-61),
  \item
    \code{death} = Number of people killed as a result of civic group
    violence (50-62).
  \item
    \code{demo} = Political regime: stable democracy, unstable democracy
    or dictatorship. Due to redundancy, the dummy variable ``unstable
    democracy'' has been left out.
  \end{itemize}
\end{itemize}

The different blocks of variables
\(\mathbf X_1, \mathbf X_2, \mathbf X_3\) are arranged in the list
format.

\footnotesize

\begin{CodeChunk}
\begin{CodeInput}
R> A <- list(
+   Agriculture = Russett[, c("gini", "farm", "rent")],
+   Industrial = Russett[, c("gnpr", "labo")],
+   Politic = Russett[, c("inst", "ecks",  "death", "demostab", "dictator")])
R> 
R> lab <- factor(
+   apply(Russett[, 9:11], 1, which.max),
+   labels = c("demost", "demoinst", "dict")
+ )
\end{CodeInput}
\end{CodeChunk}

\normalsize

\textbf{Preprocessing.} In general, and especially for the
covariance-based criterion, the data blocks might be preprocessed to
ensure comparability between variables and blocks. In order to ensure
comparability between variables, standardization is applied (zero mean
and unit variance). Such preprocessing is reached by setting the
\code{scale} argument to \code{TRUE} (default value) in the
\code{rgcca()} function. A possible strategy to make blocks comparable
is to standardize the variables and divide each block by the square root
of its number of variables \citep{Westerhuis1998}. This two-step
procedure leads to \(\mathrm{tr}(\mathbf X_j^\top \mathbf X_j )=n\) for
each block (i.e., the sum of the eigenvalues of the covariance matrix of
\(\mathbf X_j\) is equal to \(1\) whatever the block). Such
preprocessing is reached by setting the \code{scale\_block} argument to
\code{TRUE} or \code{"inertia"} (default value) in the \code{rgcca()}
function. If \code{scale\_block = "lambda1"}, each block is divided by
the square root of the highest eigenvalue of its empirical covariance
matrix. If standardization is applied (\code{scale = TRUE}), the block
scaling is applied on the result of the standardization.

\textbf{Definition of the design matrix} \(\mathbf{C}\). From Russett's
hypotheses, it is difficult for a country to escape dictatorship when
agricultural inequality is above average and industrial development is
below average. These hypotheses on the relationships between blocks are
encoded through the design matrix \(\mathbf{C}\); usually \(c_{jk} = 1\)
for two connected blocks and \(0\) otherwise. Therefore, we have decided
to connect \(\mathbf X_1\) to \(\mathbf X_3\) (\(c_{13} = 1\)),
\(\mathbf X_2\) to \(\mathbf X_3\) (\(c_{23} = 1\)) and to not connect
\(\mathbf X_1\) to \(\mathbf X_2\) (\(c_{12} = 0\)). The resulting
design matrix \(\mathbf{C}\) is:

\footnotesize

\begin{CodeChunk}
\begin{CodeInput}
R> C <- matrix(c(0, 0, 1,
+               0, 0, 1,
+               1, 1, 0), 3, 3)
R> 
R> C
\end{CodeInput}
\begin{CodeOutput}
     [,1] [,2] [,3]
[1,]    0    0    1
[2,]    0    0    1
[3,]    1    1    0
\end{CodeOutput}
\end{CodeChunk}

\normalsize

\textbf{Choice of the scheme function g}. Typical choices of scheme
functions are \(g(x) = x, x^2\), or \(\vert x \vert\). According to
\cite{VandeGeer1984}, a fair model is a model where all blocks
contribute equally to the solution in opposition to a model dominated by
only a few of the \(J\) sets. If fairness is a major objective, the user
must choose \(m=1\). \(m>1\) is preferable if the user wants to
discriminate between blocks. In practice, \(m\) is equal to \(1\), \(2\)
or \(4\). The higher the value of \(m\), the more the method acts as
block selector \citep{Tenenhaus2017}.

RGCCA using the pre-defined design matrix \(\mathbf{C}\), the factorial
scheme (\(g(x) = x^2\)), \(\tau = 1\) for all blocks (full covariance
criterion) and a number of (orthogonal) components equal to \(2\) for
all blocks is obtained by specifying appropriately the arguments
\code{connection}, \code{scheme}, \code{tau}, \code{ncomp},
\code{comp\_orth} in \code{rgcca()}. \code{verbose} (default value =
\code{TRUE}) indicates that the progress will be reported while
computing and that a plot illustrating the convergence of the algorithm
will be displayed.

\footnotesize

\begin{CodeChunk}
\begin{CodeInput}
R> fit <- rgcca(blocks = A, connection = C,
+              tau = 1, ncomp = 2,
+              scheme = "factorial",
+              scale = TRUE,
+              scale_block = FALSE,
+              comp_orth = TRUE,
+              verbose = FALSE)
\end{CodeInput}
\end{CodeChunk}

\normalsize

\textbf{The summary() and plot() functions}. The \code{summary()}
function allows summarizing the RGCCA analysis.

\footnotesize

\begin{CodeChunk}
\begin{CodeInput}
R> summary(fit)
\end{CodeInput}
\begin{CodeOutput}
Call: method='rgcca', superblock=FALSE, scale=TRUE, scale_block=FALSE, init='svd',
bias=TRUE, tol=1e-08, NA_method='na.ignore', ncomp=c(2,2,2), response=NULL,
comp_orth=TRUE 
There are J = 3 blocks.
The design matrix is:
            Agriculture Industrial Politic
Agriculture           0          0       1
Industrial            0          0       1
Politic               1          1       0

The factorial scheme is used.
Sum_{j,k} c_jk g(cov(X_j a_j, X_k a_k) = 7.9469 

The regularization parameter used for Agriculture is: 1
The regularization parameter used for Industrial is: 1
The regularization parameter used for Politic is: 1
\end{CodeOutput}
\end{CodeChunk}

\normalsize

The block-weight vectors solution of the optimization problem
(\ref{optim_RGCCA}) are available as an output of the \code{rgcca()}
function in \code{fit\$a} and correspond exactly to the weight vectors
reported in Figure 5 of \cite{Tenenhaus2011}. It is possible to display
specific block-weight vector(s) (\code{type = "weight"}) block-loadings
vector(s) (\code{type = "loadings"}) using the generic \code{plot()}
function and specifying the arguments \code{block} and \code{comp}
accordingly. The \({ \mathbf a_j^{(k)}}^\star\), mentioned in Section
\hyperref[higher-level-rgcca-algorithm]{Higher level RGCCA algorithm},
are available in \code{fit\$astar}.

\footnotesize

\begin{CodeChunk}
\begin{CodeInput}
R> plot(fit, type = "weight", block = 1:3, comp = 1,
+      display_order = FALSE, cex = 2)
\end{CodeInput}
\begin{figure}[H]

{\centering \includegraphics{figures/fig-weight-1} 

}

\caption[Block-weight vectors of a fitted RGCCA model]{Block-weight vectors of a fitted RGCCA model.}\label{fig:fig-weight}
\end{figure}
\end{CodeChunk}

\normalsize

As a component-based method, the \pkg{RGCCA} package provides block
components as an output of the \code{rgcca()} function in \code{fit\$Y}.
Various graphical representations of the block components are available
using the \code{plot()} function on a fitted RGCCA object by setting the
\code{type} argument. They include factor plot (\code{"sample"}),
correlation circle (\code{"cor_circle"}), or biplot (\code{"biplot"}).
This graphical display allows visualization of the sources of
variability within blocks, the relationships between variables within
and between blocks, and the correlation between blocks. The factor plot
of the countries obtained by crossing the block components associated
with the agricultural inequality and industrial development blocks,
marked by their political regime in 1960, is shown below.

\footnotesize

\begin{CodeChunk}
\begin{CodeInput}
R> plot(fit, type = "sample",
+      block = 1:2, comp = 1,
+      resp = lab, repel = TRUE, cex = 2)
\end{CodeInput}
\begin{figure}[H]

{\centering \includegraphics{figures/fig-sample1-1} 

}

\caption{\label{fig:sample}Graphical display of the countries by drawing the block component of the first block against the block component of the second block, colored according to their political regime.}\label{fig:fig-sample1}
\end{figure}
\end{CodeChunk}

\normalsize

Countries aggregate together when they share similarities. It may be
noted that the lower right quadrant concentrates on dictatorships. It is
difficult for a country to escape dictatorship when its industrial
development is below average and its agricultural inequality is above
average. It is worth pointing out that some unstable democracies located
in this quadrant (or close to it) became dictatorships for a period of
time after 1960: Greece (1967-1974), Brazil (1964-1985), Chili
(1973-1990), and Argentina (1966-1973).

The AVEs of the different blocks are reported in the axes of Figure
\ref{fig:sample}. All AVEs (defined in \eqref{AVE_X}-\eqref{AVE_inner})
are available as output of the \code{rgcca()} function in
\code{fit\$AVE}. These indicators of model quality can also be
visualized using the generic \code{plot()} function.

\footnotesize

\begin{CodeChunk}
\begin{CodeInput}
R> plot(fit, type = "ave", cex = 2)
\end{CodeInput}
\begin{figure}[H]

{\centering \includegraphics{figures/fig-ave-1} 

}

\caption[Average variance explained of the different blocks]{Average variance explained of the different blocks.}\label{fig:fig-ave}
\end{figure}
\end{CodeChunk}

\normalsize

The strength of the relations between each block component and each
variable can be visualized using correlation circles or biplot
representations.

\footnotesize

\begin{CodeChunk}
\begin{CodeInput}
R> plot(fit, type = "cor_circle", block = 1, comp = 1:2, 
+      display_blocks = 1:3, cex = 2)
\end{CodeInput}
\begin{figure}[H]

{\centering \includegraphics{figures/fig-cor-circle-1} 

}

\caption[Correlation circle associated with the first two components of the first block]{Correlation circle associated with the first two components of the first block.}\label{fig:fig-cor-circle}
\end{figure}
\end{CodeChunk}

\normalsize

By default, all the variables are displayed on the correlation circle.
However, it is possible to choose the block(s) to display
(\code{display\_blocks}) in the correlation\_circle.

\footnotesize

\begin{CodeChunk}
\begin{CodeInput}
R> plot(fit, type = "biplot", block = 1, 
+      comp = 1:2, repel = TRUE, 
+      resp = lab, cex = 2,
+      show_arrow = TRUE)
\end{CodeInput}
\begin{figure}[H]

{\centering \includegraphics{figures/fig-biplot1-1} 

}

\caption[Biplot associated with the first two components of the first block]{Biplot associated with the first two components of the first block.}\label{fig:fig-biplot1}
\end{figure}
\end{CodeChunk}

\normalsize

As we will see in the next section when the superblock option is
considered (\code{superblock = TRUE} or \code{method} set to a method
that induces the use of superblock), global components can be derived.
The space spanned by the global components can be viewed as a consensus
space that integrates all the modalities and facilitates the
visualization of the results and their interpretation.

\textbf{Bootstrap confidence intervals}. We illustrate the use of the
\code{rgcca\_bootstrap()} function. It is possible to set the number of
bootstrap samples using the \code{n\_boot} argument:

\footnotesize

\begin{CodeChunk}
\begin{CodeInput}
R> set.seed(0)
R> boot_out <- rgcca_bootstrap(fit, n_boot = 500, n_cores = 1)
\end{CodeInput}
\end{CodeChunk}

\normalsize

The bootstrap results are detailed using the \code{summary()} function,

\footnotesize

\begin{CodeChunk}
\begin{CodeInput}
R> summary(boot_out, block = 1:3, ncomp = 1)
\end{CodeInput}
\begin{CodeOutput}
Call: method='rgcca', superblock=FALSE, scale=TRUE, scale_block=FALSE, init='svd',
bias=TRUE, tol=1e-08, NA_method='na.ignore', ncomp=c(2,2,2), response=NULL,
comp_orth=TRUE 
There are J = 3 blocks.
The design matrix is:
            Agriculture Industrial Politic
Agriculture           0          0       1
Industrial            0          0       1
Politic               1          1       0

The factorial scheme is used.

Extracted statistics from 500 bootstrap samples.
Block-weight vectors for component 1: 
         estimate    mean     sd lower_bound upper_bound bootstrap_ratio   pval
gini       0.6602  0.6360 0.0696      0.4744       0.725           9.490 0.0000
farm       0.7445  0.7304 0.0555      0.6443       0.828          13.423 0.0000
rent       0.0994  0.0762 0.2203     -0.3943       0.454           0.451 0.4663
gnpr       0.6891  0.6894 0.0298      0.6252       0.739          23.140 0.0000
labo      -0.7247 -0.7232 0.0278     -0.7804      -0.673         -26.087 0.0000
inst       0.1692  0.1672 0.1174     -0.0801       0.357           1.442 0.0917
ecks       0.4418  0.4340 0.0592      0.3224       0.545           7.458 0.0000
death      0.4784  0.4699 0.0483      0.3769       0.560           9.914 0.0000
demostab  -0.5574 -0.5520 0.0509     -0.6400      -0.432         -10.942 0.0000
dictator   0.4864  0.4831 0.0524      0.3846       0.586           9.285 0.0000
         adjust.pval
gini           0.000
farm           0.000
rent           0.523
gnpr           0.000
labo           0.000
inst           0.131
ecks           0.000
death          0.000
demostab       0.000
dictator       0.000
\end{CodeOutput}
\end{CodeChunk}

\normalsize

and displayed using the \code{plot()} function.

\footnotesize

\begin{CodeChunk}
\begin{CodeInput}
R> plot(boot_out, type = "weight", 
+      block = 1:3, comp = 1, 
+      display_order = FALSE, cex = 2,
+      show_stars = TRUE)
\end{CodeInput}
\begin{figure}[H]

{\centering \includegraphics{figures/fig-boot1-1} 

}

\caption[Bootstrap confidence intervals for the block-weight vectors]{Bootstrap confidence intervals for the block-weight vectors.}\label{fig:fig-boot1}
\end{figure}
\end{CodeChunk}

\normalsize

Each weight is shown along with its associated bootstrap confidence
interval. If \code{show\_stars = TRUE}, stars reflecting the p-value of
assigning a strictly positive or negative weight to this variable are
displayed.

\textbf{Choice of the shrinkage parameters}. In the previous section,
the shrinkage parameters were manually set. However, the \pkg{RGCCA}
package introduces three fully automated strategies for selecting the
optimal shrinkage parameters.

\textbf{The Schafer and Strimmer analytical formula.} For each block
\(j\), an ``optimal'' shrinkage parameter \(\tau_j\) can be obtained
using the Schafer and Strimmer analytical formula \citep{Schafer2005} by
setting the \code{tau} argument of the \code{rgcca()} function to
\code{"optimal"}.

\footnotesize

\begin{CodeChunk}
\begin{CodeInput}
R> fit <- rgcca(blocks = A, connection = C,
+              tau = "optimal", scheme = "factorial")
\end{CodeInput}
\end{CodeChunk}

\normalsize

The optimal shrinkage parameters are given by:

\footnotesize

\begin{CodeChunk}
\begin{CodeInput}
R> fit$call$tau
\end{CodeInput}
\begin{CodeOutput}
[1] 0.08853216 0.02703256 0.08422566
\end{CodeOutput}
\end{CodeChunk}

\normalsize

This automatic estimation of the shrinkage parameters allows one to come
closer to the correlation criterion, even in the case of high
multicollinearity or when the number of individuals is smaller than the
number of variables.

As before, all the fitted RGCCA objects can be visualized/bootstrapped
using the \code{summary()}, \code{plot()} and \code{rgcca\_bootstrap()}
functions.

\textbf{Permutation strategy.} We illustrate the use of the
\code{rgcca\_permutation()} function here. The number of permutations
can be set using the \code{n\_perms} argument:

\footnotesize

\begin{CodeChunk}
\begin{CodeInput}
R> set.seed(0)
R> perm_out <- rgcca_permutation(blocks = A, connection = C,
+                               par_type = "tau",
+                               par_length = 10,
+                               n_cores = 1,
+                               n_perms = 10)
\end{CodeInput}
\end{CodeChunk}

\normalsize

By default, the \code{rgcca\_permutation} function generates 10 sets of
tuning parameters uniformly between some minimal values (0 for RGCCA and
\(1/\text{sqrt}(\text{ncol})\) for SGCCA) and 1. Results of the
permutation procedure are summarized using the generic \code{summary()}
function,

\footnotesize

\begin{CodeChunk}
\begin{CodeInput}
R> summary(perm_out)
\end{CodeInput}
\begin{CodeOutput}
Call: method='rgcca', superblock=FALSE, scale=TRUE, scale_block=TRUE, init='svd',
bias=TRUE, tol=1e-08, NA_method='na.ignore', ncomp=c(1,1,1), response=NULL,
comp_orth=TRUE 
There are J = 3 blocks.
The design matrix is:
            Agriculture Industrial Politic
Agriculture           0          0       1
Industrial            0          0       1
Politic               1          1       0

The factorial scheme is used.

Tuning parameters (tau) used: 
   Agriculture Industrial Politic
1        1.000      1.000   1.000
2        0.889      0.889   0.889
3        0.778      0.778   0.778
4        0.667      0.667   0.667
5        0.556      0.556   0.556
6        0.444      0.444   0.444
7        0.333      0.333   0.333
8        0.222      0.222   0.222
9        0.111      0.111   0.111
10       0.000      0.000   0.000

   Tuning parameters Criterion Permuted criterion     sd zstat p-value
1     1.00/1.00/1.00     0.708             0.0909 0.0479 12.88       0
2     0.89/0.89/0.89     0.758             0.0993 0.0512 12.86       0
3     0.78/0.78/0.78     0.814             0.1093 0.0549 12.83       0
4     0.67/0.67/0.67     0.878             0.1214 0.0592 12.80       0
5     0.56/0.56/0.56     0.953             0.1365 0.0640 12.76       0
6     0.44/0.44/0.44     1.040             0.1561 0.0696 12.70       0
7     0.33/0.33/0.33     1.144             0.1829 0.0764 12.58       0
8     0.22/0.22/0.22     1.273             0.2233 0.0854 12.29       0
9     0.11/0.11/0.11     1.449             0.2974 0.1007 11.44       0
10    0.00/0.00/0.00     1.934             0.6049 0.1419  9.37       0
The best combination is: 1.00/1.00/1.00 for a z score of 12.9 and a p-value of 0
\end{CodeOutput}
\end{CodeChunk}

\normalsize

and displayed using the generic \code{plot()} function.

\footnotesize

\begin{CodeChunk}
\begin{CodeInput}
R> plot(perm_out, cex = 2)
\end{CodeInput}
\begin{figure}[H]

{\centering \includegraphics{figures/fig-permutation-1} 

}

\caption[Values of the objective function of RGCCA against the sets of tuning parameters, triangles correspond to evaluations on non-permuted datasets]{Values of the objective function of RGCCA against the sets of tuning parameters, triangles correspond to evaluations on non-permuted datasets.}\label{fig:fig-permutation}
\end{figure}
\end{CodeChunk}

\normalsize

The fitted permutation object, \code{perm\_out}, can be directly
provided as the output of \code{rgcca()} and visualized/bootstrapped as
usual.

\footnotesize

\begin{CodeChunk}
\begin{CodeInput}
R> fit <- rgcca(perm_out)
\end{CodeInput}
\end{CodeChunk}

\normalsize

Of course, it is possible to define explicitly the combination of
regularization parameters to be tested. In that case, a matrix of
dimension \(K \times J\) is required. Each row of this matrix
corresponds to one set of tuning parameters. Alternatively, a numeric
vector of length \(J\) indicating the maximum range values to be tested
can be given. The set of parameters is then uniformly generated between
the minimum values (0 for RGCCA and \(1/\text{sqrt}(\text{ncol})\) for
SGCCA) and the maximum values specified by the user with
\code{par\_value}.

\textbf{Cross-validation strategy.} The optimal tuning parameters can
also be obtained by cross-validation. In the forthcoming section, we
will illustrate this strategy in the second case study.

\textbf{RGCCA with a superblock}. To conclude this section on the
analysis of the Russett dataset, we consider multiple co-inertia
analysis \citep{Chessel1996}
\citep[MCOA, also called MCIA in][]{Cantini2021} with \(2\) components
per block.

See \code{available\_methods()} for a list of pre-specified multiblock
component methods.

\footnotesize

\begin{CodeChunk}
\begin{CodeInput}
R> fit.mcoa <- rgcca(blocks = A, method = "mcoa", ncomp = 2)
\end{CodeInput}
\end{CodeChunk}

\normalsize

Interestingly, the \code{summary()} function reports the arguments
implicitly specified to perform MCOA.

\footnotesize

\begin{CodeChunk}
\begin{CodeInput}
R> summary(fit.mcoa)
\end{CodeInput}
\begin{CodeOutput}
Call: method='mcoa', superblock=TRUE, scale=TRUE, scale_block='inertia', init='svd',
bias=TRUE, tol=1e-08, NA_method='na.ignore', ncomp=c(2,2,2,2), response=NULL,
comp_orth=FALSE 
There are J = 4 blocks.
The design matrix is:
            Agriculture Industrial Politic superblock
Agriculture           0          0       0          1
Industrial            0          0       0          1
Politic               0          0       0          1
superblock            1          1       1          0

The factorial scheme is used.
Sum_{j,k} c_jk g(cov(X_j a_j, X_k a_k) = 3.578 

The regularization parameter used for Agriculture is: 1
The regularization parameter used for Industrial is: 1
The regularization parameter used for Politic is: 1
The regularization parameter used for superblock is: 0
\end{CodeOutput}
\end{CodeChunk}

\normalsize

It is possible to display specific output as previously using the
generic \code{plot()} function by specifying the argument \code{type}
accordingly. MCOA enables individuals to be represented in the space
spanned by the first global components. The biplot representation
associated with this consensus space is given below.

\footnotesize

\begin{CodeChunk}
\begin{CodeInput}
R> plot(fit.mcoa, type = "biplot", 
+      block = 4, comp = 1:2, 
+      response = lab, 
+      repel = TRUE, cex = 2)
\end{CodeInput}
\begin{figure}[H]

{\centering \includegraphics{figures/fig-biplot2-1} 

}

\caption[Biplot of the countries obtained by crossing the two first components of the superblock]{Biplot of the countries obtained by crossing the two first components of the superblock. Individuals are colored according to their political regime and variables according to their block membership.}\label{fig:fig-biplot2}
\end{figure}
\end{CodeChunk}

\normalsize

As previously, this model can be easily bootstrapped using the
\code{rgcca\_bootstrap()} function, and the bootstrap confidence
intervals are still available using the \code{summary()} and
\code{plot()} functions.

\subsection{High dimensional case study: the glioma
study}\label{high-dimensional-case-study-the-glioma-study}

\textbf{Biological problem.} Brain tumors are children's most common
solid tumors and have the highest mortality rate of all pediatric
cancers. Despite advances in multimodality therapy, children with
pediatric high-grade gliomas (pHGG) invariably have an overall survival
of around 20\% at 5 years. Depending on their location (e.g., brainstem,
central nuclei, or supratentorial), pHGG present different
characteristics in terms of radiological appearance, histology, and
prognosis. Our hypothesis is that pHGG have different genetic origins
and oncogenic pathways depending on their location. Thus, the biological
processes involved in the development of the tumor may be different from
one location to another, as has been frequently suggested.

\textbf{Description of the data.} Pretreatment frozen tumor samples were
obtained from 53 children with newly diagnosed pHGG from Necker Enfants
Malades (Paris, France) \citep{Puget2012}. The 53 tumors are divided
into 3 locations: supratentorial (HEMI), central nuclei (MIDL), and
brain stem (DIPG). The final dataset is organized into 3 blocks of
variables defined for the 53 tumors: the first block \(\mathbf{X}_1\)
provides the expression of \(15702\) genes (GE). The second block
\(\mathbf{X}_2\) contains the imbalances of \(1229\) segments (CGH) of
chromosomes. \(\mathbf{X}_3\) is a block of dummy variables describing
the categorical variable location. One dummy variable has been left out
because of redundancy with the others.

The next lines of code can be run to download the dataset from
\url{http://biodev.cea.fr/sgcca/}:

\footnotesize

\begin{CodeChunk}
\begin{CodeInput}
R> if (!("gliomaData" %in% rownames(installed.packages()))) {
+   destfile <- tempfile()
+   download.file("http://biodev.cea.fr/sgcca/gliomaData_0.4.tar.gz", destfile)
+   install.packages(destfile, repos = NULL, type = "source")
+ }
\end{CodeInput}
\end{CodeChunk}

\normalsize

\footnotesize

\begin{CodeChunk}
\begin{CodeInput}
R> data("ge_cgh_locIGR", package = "gliomaData")
R> 
R> blocks <- ge_cgh_locIGR$multiblocks
R> Loc <- factor(ge_cgh_locIGR$y)
R> levels(Loc) <- colnames(ge_cgh_locIGR$multiblocks$y)
R> blocks[[3]] <- Loc
R> 
R> vapply(blocks, NCOL, FUN.VALUE = 1L)
\end{CodeInput}
\end{CodeChunk}

\normalsize

We impose \(\mathbf{X}_1\) and \(\mathbf{X}_2\) to be connected to
\(\mathbf{X}_3\). This design is commonly used in many applications and
is oriented toward predicting the location. The argument
\code{response = 3} of the \code{rgcca()} function encodes this design.

\footnotesize

\begin{CodeChunk}
\begin{CodeInput}
R> fit.rgcca <- rgcca(blocks = blocks, response = 3, ncomp = 2, verbose = FALSE)
\end{CodeInput}
\end{CodeChunk}

\normalsize

When the response variable is qualitative, two steps are implicitly
performed: (i) disjunctive coding and (ii) the associated shrinkage
parameter is set to \(0\) regardless of the value specified by the user.

\footnotesize

\begin{CodeChunk}
\begin{CodeInput}
R> fit.rgcca$call$connection
R> fit.rgcca$call$tau
\end{CodeInput}
\end{CodeChunk}

\normalsize

\textbf{The primal/dual RGCCA algorithm}. From the dimension of each
block (\(n>p\) or \(n\leq p\)), \code{rgcca()} selects automatically the
dual formulation for \(\mathbf{X}_1\) and \(\mathbf{X}_2\) and the
primal one for \(\mathbf{X}_3\). The formulation used for each block is
returned using the following command:

\footnotesize

\begin{CodeChunk}
\begin{CodeInput}
R> fit.rgcca$primal_dual
\end{CodeInput}
\end{CodeChunk}

\normalsize

The dual formulation makes the RGCCA algorithm highly efficient, even in
a high-dimensional setting.

\footnotesize

\begin{CodeChunk}
\begin{CodeInput}
R> system.time(
+   rgcca(blocks = blocks, response = 3)
+ )
\end{CodeInput}
\end{CodeChunk}

\normalsize

RGCCA enables visual inspection of the spatial relationships between
classes. This facilitates assessment of the quality of the
classification and makes it possible to determine which components
capture the discriminant information readily.

\footnotesize

\begin{CodeChunk}
\begin{CodeInput}
R> plot(fit.rgcca, type = "sample", block = 1:2,
+      comp = 1, response = Loc, cex = 2)
\end{CodeInput}
\end{CodeChunk}

\normalsize

For easier interpretation of the results, especially in high-dimensional
settings, adding penalties promoting sparsity within the RGCCA
optimization problem is often appropriate. For that purpose, an
\(\ell_1\) penalization on the weight vectors
\(\mathbf{a}_1, \ldots, \mathbf{a}_J\) is applied. the \code{sparsity}
argument of \code{rgcca()} varies between 1/sqrt(ncol) and 1 (larger
values of \code{sparsity} correspond to less penalization) and controls
the amount of sparsity of the weight vectors
\(\mathbf{a}_1, \ldots, \mathbf{a}_J\). If \code{sparsity} is a vector,
\(\ell_1\)-penalties are the same for all the weights corresponding to
the same block but different components:

\begin{equation}
\forall h, \Vert \mathbf{a}_j^{(h)} \Vert_{\ell_1} \leq \small{\texttt{sparsity}}_j \sqrt{p_j},
\end{equation}

with \(p_j\) the number of variables of \(\mathbf X_j\).

If \code{sparsity} is a matrix, row \(h\) of \code{sparsity} defines the
constraints applied to the weights corresponding to components \(h\):

\begin{equation}
\forall h, \Vert \mathbf{a}_j^{(h)} \Vert_{\ell_1} \leq \small{\texttt{sparsity}}_{h,j} \sqrt{p_j}.
\end{equation}

\textbf{SGCCA for the Glioma dataset}. The algorithm associated with the
optimization problem (\ref{optim_SGCCA}) is available through the
\code{rgcca()} function with the argument \code{method = "sgcca"} or by
specifying directly the \code{sparsity} argument as below.

\footnotesize

\begin{CodeChunk}
\begin{CodeInput}
R> fit.sgcca <- rgcca(blocks = blocks, response = 3, ncomp = 2,
+                    sparsity = c(0.0710, 0.2000, 1),
+                    verbose = FALSE)
\end{CodeInput}
\end{CodeChunk}

\normalsize

The \code{summary()} function allows summarizing the SGCCA analysis,

\footnotesize

\begin{CodeChunk}
\begin{CodeInput}
R> summary(fit.sgcca)
\end{CodeInput}
\end{CodeChunk}

\normalsize

and the \code{plot()} function returns the same graphical displays as
RGCCA. We skip these representations for the sake of brevity.

\textbf{Determining the sparsity parameters by cross-validation}. Of
course, it is still possible to determine the optimal sparsity
parameters by permutation. This is made possible by setting the
\code{par\_type} argument to \code{"sparsity"} (instead of \code{"tau"})
within the \code{rgcca\_permutation()} function. However, in this
section, we will adopt a different approach. The optimal tuning
parameters can be determined by cross-validating using different
indicators of quality for classification and regression.

This cross-validation protocol is made available through the
\code{rgcca\_cv} function and is used here to predict the tumors'
location. In this situation, the goal is to maximize the cross-validated
balanced accuracy (\code{metric = "Balanced\_Accuracy"}) in a model
where we try to predict the response block from all the block components
with a user-defined classifier (\code{prediction\_model = "lda"}). The
cross-validation protocol is set by specifying the arguments like the
number of folds (\code{k}) and the number of cross-validations to be run
(\code{n\_run}). Also, we decide to upper bound the sparsity parameters
for \(\mathbf X_1\) and \(\mathbf X_2\) to \(0.2\) to achieve an
attractive amount of sparsity.

\footnotesize

\begin{CodeChunk}
\begin{CodeInput}
R> set.seed(0) 
R> in_train <- caret::createDataPartition(
+   blocks[[3]], p = .75, list = FALSE
+ )
R> training <- lapply(blocks, function(x) as.matrix(x)[in_train, , drop = FALSE])
R> testing <- lapply(blocks, function(x) as.matrix(x)[-in_train, , drop = FALSE])
R> 
R> cv_out <- rgcca_cv(blocks = training, response = 3,
+                    par_type = "sparsity",
+                    par_value = c(.2, .2, 0),
+                    par_length = 10,
+                    prediction_model = "lda",
+                    validation = "kfold",
+                    k = 7, n_run = 3, metric = "Balanced_Accuracy",
+                    n_cores = 1)
\end{CodeInput}
\end{CodeChunk}

\normalsize

\code{rgcca\_cv()} relies on the \pkg{caret} package. As a direct
consequence, an astonishingly large number of models are made available
(see \code{caret::modelLookup()}). Results of the cross-validation
procedure are reported using the generic \code{summary()} function,

\footnotesize

\begin{CodeChunk}
\begin{CodeInput}
R> summary(cv_out)
\end{CodeInput}
\end{CodeChunk}

\normalsize

and displayed using the generic \code{plot()} function.

\footnotesize

\begin{CodeChunk}
\begin{CodeInput}
R> plot(cv_out, cex = 2)
\end{CodeInput}
\end{CodeChunk}

\normalsize

As before, the optimal sparsity parameters can be used to fit a new
model, and the resulting optimal model can be visualized/bootstrapped.

\footnotesize

\begin{CodeChunk}
\begin{CodeInput}
R> fit <- rgcca(cv_out)
R> summary(fit)
\end{CodeInput}
\end{CodeChunk}

\normalsize

Note that the sparsity parameter associated with \(\mathbf{X}_3\)
switches automatically to \(\tau_3 = 0\). This choice is justified by
the fact that we were not looking for a block component \(\mathbf y_3\)
that explained its own block well (since \(\mathbf{X}_3\) is a group
coding matrix) but one that is correlated with its neighboring
components.

At last, \code{rgcca\_predict()} can be used for predicting new blocks,

\footnotesize

\begin{CodeChunk}
\begin{CodeInput}
R> pred <- rgcca_predict(fit, blocks_test = testing, prediction_model = "lda")
\end{CodeInput}
\end{CodeChunk}

\normalsize

and a \pkg{caret} summary of the performances can be reported.

\footnotesize

\begin{CodeChunk}
\begin{CodeInput}
R> pred$confusion$test
\end{CodeInput}
\end{CodeChunk}

\normalsize

The \code{rgcca\_transform()} function can be used if, for a specific
reason, only the block components are wanted for the test set.

\footnotesize

\begin{CodeChunk}
\begin{CodeInput}
R> projection <- rgcca_transform(fit, blocks_test = testing)
\end{CodeInput}
\end{CodeChunk}

\normalsize

\textbf{Stability selection}. We finally illustrate the use of the
\code{rgcca\_stability()} function:

\footnotesize

\begin{CodeChunk}
\begin{CodeInput}
R> set.seed(0) 
R> fit_stab <- rgcca_stability(fit,
+                             keep = vapply(
+                               fit$a, function(x) mean(x != 0),
+                               FUN.VALUE = 1.0
+                             ),
+                             n_boot = 100, verbose = TRUE, n_cores = 1)
\end{CodeInput}
\end{CodeChunk}

\normalsize

Once the most stable variables have been found, a new model using these
variables is automatically fitted. This last model can be visualized
using the usual \code{summary()} and \code{plot()} functions. We can
finally apply the bootstrap procedure on the most stable variables.

\footnotesize

\begin{CodeChunk}
\begin{CodeInput}
R> set.seed(0)
R> boot_out <- rgcca_bootstrap(fit_stab, n_boot = 500)
\end{CodeInput}
\end{CodeChunk}

\normalsize

The bootstrap results can be visualized using the generic \code{plot()}
function. We use the \code{n\_mark} parameter to display the top 50
variables of GE.

\footnotesize

\begin{CodeChunk}
\begin{CodeInput}
R> plot(boot_out, block = 1,
+      display_order = FALSE,
+      n_mark = 50, cex = 1.5, cex_sub = 17,
+      show_star = TRUE)
\end{CodeInput}
\end{CodeChunk}

\normalsize

\section{Conclusion}\label{conclusion}

The RGCCA framework gathers sixty years of multiblock component methods
and offers a unified implementation strategy for these methods. The
\pkg{RGCCA} package is available on the Comprehensive \proglang{R}
Archive Network (CRAN) and GitHub
\url{https://github.com/rgcca-factory/RGCCA}. This release of the
\pkg{RGCCA} package includes:

\begin{itemize}
\item Several strategies for determining the shrinkage parameters/level of sparsity automatically: Schaffer \& Strimmer's analytical formulae, cross-validation, or permutation strategy.

\item A bootstrap resampling procedure for assessing the reliability of the parameter estimates of S/RGCCA.

\item Dedicated functions for graphical displays of the output of RGCCA (sample plot, correlation circle, biplot, ...).

\item Various deflation strategies for obtaining orthogonal block-components or orthogonal block-weight vectors.


\item Strategies for handling missing data. Specifically, multiblock data faces two types of missing data structure: (i) if an observation $i$ has missing values on a whole block $j$ and (ii) if an observation $i$ has some missing values on a block $j$ (but not all). For these two situations, we exploit the algorithmic solution proposed for PLS path modeling to deal with missing data \citep[see][]{Tenenhaus2005}.

\item Special attention has been paid to providing a bunch of "mathematical" unit tests which, in a sense, guarantee the implementation quality. Also, when appropriate, a particular focus was given to recovering the results of other \proglang{R} packages of the literature, including \pkg{ade4} and \pkg{FactoMineR}.
\end{itemize}

The \pkg{RGCCA} package will be a valuable resource for researchers and
practitioners who are interested in multiblock data analysis to gain new
insights and improve decision-making.

The RGCCA framework is constantly evolving and extending. Indeed, we
proposed RGCCA for multigroup data \citep{Tenenhaus2014b}, RGCCA for
multiway data \citep{Gloaguen2020, Girka2023}, and RGCCA for (sparse and
irregular) functional data \citep{Sort2023}. In addition, maximizing
successive criteria may be sub-optimal from an optimization point of
view, where a single global criterion is preferred. A global version of
RGCCA \citep{Gloaguen2020b}, which allows simultaneously extracting
several components per block (no deflation procedure required), has been
proposed. Also, it is possible to use RGCCA in structural equation
modeling with latent and emergent variables for obtaining consistent and
asymptotically normal estimators of the parameters
\citep{Tenenhaus2023}. At last, several alternatives for handling
missing values are discussed in \cite{Peltier2022}. Work in progress
includes the integration of all these novel approaches in the next
release of the \pkg{RGCCA} package. The modular design of the
\code{rgcca()} function will greatly simplify the integration of these
extensions into the package.

\section*{Acknowledgments}\label{acknowledgments}
\addcontentsline{toc}{section}{Acknowledgments}

This project has received funding from UDOPIA - ANR-20-THIA-0013, the
European Union's Horizon 2020 research and innovation program under
grant agreement No 874583 and the AP-HP Foundation, within the framework
of the AIRACLES Chair.

\renewcommand\refname{References}
\bibliography{biblio.bib}



\end{document}
